%!TEX root = htm.tex
\section{Hybrid transactional memory (HyTM)}
\label{sec:hytm}
%
In this section, we extend the formal model of HyTMs originally proposed in \cite{htmdisc15} to accomodate non-speculative accesses within hardware transactions.
The resulting model is sufficiently expressive to capture the executions of existing hardware transactions like \emph{Intel Haswell}~\cite{Rei12},
\emph{IBM Power8}~\cite{bluegene} and \emph{AMD ASF}~\cite{asf}.

\paragraph{Transactional memory (TM).} 
A \emph{transaction} is a sequence of \emph{transactional operations}
(or \emph{t-operations}), reads and writes, performed on a set of \emph{transactional objects} 
(\emph{t-objects}). 
A TM \emph{implementation} provides a set of
concurrent \emph{processes} with deterministic algorithms that implement reads and
writes on t-objects using  a set of \emph{base objects}.
More precisely, for each transaction $T_k$, a TM implementation must support the following t-operations: 
$\mathit{read}_k(X)$, where $X$ is a t-object, that returns a value in
a domain $V$
or a special value $A_k\notin V$ (\emph{abort}),
$\mathit{write}_k(X,v)$, for a value $v \in V$,
that returns $\mathit{ok}$ or $A_k$, and
$\mathit{tryC}_k$ that returns $C_k\notin V$ (\emph{commit}) or $A_k$.
Additionally, we assume that a transaction $T_k$
may perform a $\mathit{start_k}$ that is the first t-operation performed by $T_k$ prior to invoking any $\mathit{read}_k$ or $\mathit{write}_k$.

\paragraph{Configurations and executions.} 
A \emph{configuration} of a TM implementation specifies the state of each base object and each process. 
In the \emph{initial} configuration, each base object has its initial value and each process is in its initial state. 
An \emph{event} (or \emph{step}) of a transaction invoked by some process is an invocation of a t-operation, 
a response of a t-operation, or an atomic \emph{primitive} operation applied to base object along with its response. 
%An event $e$ is \emph{applicable} to a configuration $C$ if $e$ can legally be applied to $C$. 
%Applying an event $e$ to a configuration $C$ results in another configuration $C' = e(C)$.
An \emph{execution fragment} is a (finite or infinite) sequence of events $E = e_1,e_2,\dots$. 
%$E$ is applicable to a configuration $C$ if $e_1$ is applicable to $C$, $e_2$ is applicable to $e_1(C)$, 
%and so forth. 
An \emph{execution} of a TM implementation $\mathcal{M}$ is an
execution fragment where, informally, each event respects the
specification of base objects and the algorithms specified by $\mathcal{M}$.

%applicable to the initial configuration 
%where each event of each transaction is issued according to $\mathcal{M}$.
The \emph{read set} (resp., the \emph{write set}) of a transaction $T_k$ in an execution $E$,
denoted $\Rset_E(T_k)$ (and resp. $\Wset_E(T_k)$), is the set of t-objects that $T_k$ attempts to read (and resp. write) 
by issuing a t-read (and resp. t-write) invocation in $E$ (for brevity, we sometimes 
omit the subscript $E$ from the notation).
The \emph{data set} of $T_k$ is $\Dset(T_k)=\Rset(T_k)\cup\Wset(T_k)$.
$T_k$ is called \emph{read-only} if $\Wset(T_k)=\emptyset$; \emph{write-only} if $\Rset(T_k)=\emptyset$ and
\emph{updating} if $\Wset(T_k)\neq\emptyset$.

For any finite execution $E$ and execution fragment $E'$, $E\cdot E'$ denotes the concatenation of $E$ and $E'$
and we say that $E\cdot E'$ is an \emph{extension}
of $E$.
%Let $E$ be an execution fragment.
For every transaction identifier $k$,
$E|k$ denotes the subsequence of $E$ restricted to events of
transaction $T_k$.
If $E|k$ is non-empty,
we say that $T_k$ \emph{participates} in $E$,
and let $\txns(E)$ denote the set of transactions that participate in $E$.
Two executions $E$ and $E'$
are \emph{indistinguishable} to a set $\mathcal{T}$ of transactions, if
for each transaction $T_k \in \mathcal{T}$, $E|k=E'|k$.

A transaction $T_k\in \txns(E)$ is \emph{complete in $E$} if
$E|k$ ends with a response event.
The execution $E$ is \emph{complete} if all transactions in $\txns(E)$
are complete in $E$.
A transaction $T_k\in \txns(E)$ is \emph{t-complete} if $E|k$
ends with $A_k$ or $C_k$; otherwise, $T_k$ is \emph{t-incomplete}.

We assume that base objects are accessed with \emph{read-modify-write} (rmw) primitives. 
A rmw primitive $\langle g,h \rangle$ applied to a base object 
atomically updates the value of the object with a new value, which is
a function $g(v)$ of the old value $v$, and returns a response $h(v)$.
%A rmw primitive is \emph{trivial} if it never affects the value of a
%base object, otherwise it is \emph{nontrivial}.
A rmw primitive event on a base object is \emph{trivial} if, in any configuration, its application
does not change the state of the object. 
Otherwise, it is called \emph{nontrivial}.
Events $e$ and $e'$ of an execution $E$  \emph{contend} on a base
object $b$ if they are both primitives on $b$ in $E$ and at least 
one of them is nontrivial.

\paragraph{Hybrid transactional memory executions.}
We now describe the execution model of a \emph{Hybrid transactional memory (HyTM)} implementation.
In our HyTM model, shared memory configurations may be modified by accessing base objects via two kinds of
primitives: \emph{direct} and \emph{cached}.
(i) In a direct access, the rmw primitive operates on the memory state:
the direct-access event atomically reads the value of the object in
the shared memory and, if necessary, modifies it.
(ii) In a cached access performed by a process $i$, the rmw primitive operates on the \emph{cached}
state recorded in process $i$'s \emph{tracking set} $\tau_i$ (analogous to \emph{L1 cache} of process $i$). 

More precisely, $\tau_i$ is a set of triples $(b, v, m)$ where $b$ is a base object identifier, $v$ is a value, 
and $m \in \{\shared, \exclusive\}$ is an access \emph{mode}. 
The triple $(b, v, m)$ is added to the tracking set when $i$ performs a cached
rmw access of $b$, where $m$ is set to $\exclusive$ if the access is
nontrivial, and to $\shared$ otherwise.  
%we assume that there exists some constant $\TS$ (representing the size of the L1 cache)
% such that the condition $|\tau_i| \leq \TS$ must always hold; this
% condition will be enforced by our model.
A base object $b$ is \emph{present} in $\tau_i$ with mode $m$ if $\exists v, (b,v,m) \in \tau_i$.

A trivial (resp.\ nontrivial) 
cached primitive $\langle g,h \rangle$ applied to $b$ 
by process $i$ 
% first checks the condition $|\tau_i|=\TS$ and if so, it
% sets $\tau_i=\emptyset$ and immediately returns $\bot$ (we call this event a
% \emph{capacity abort}). 
% We assume that $\TS$ is large enough so that no transaction 
% with data set of size $1$ can incur a capacity abort.
% %
% If the transaction does not incur a capacity abort, the process 
checks whether $b$ is present in exclusive
(resp.\ any) mode in $\tau_j$ 
for any $j\neq i$. If so, $\tau_i$ is set to $\emptyset$ and the
primitive returns $\bot$. 
%
Otherwise, the triple $(b, v, \shared)$ (resp. $(b, g(v), \exclusive)$)
is added to $\tau_i$,  where $v$ is the most recent cached value of $b$ in $\tau_i$
(in case $b$ was previously accessed by $i$ within the current
hardware transaction) or the value of $b$ in the current
memory configuration, and finally $h(v)$ is returned.
%

\paragraph{Hardware aborts.}
A tracking set can be \emph{invalidated} by a concurrent process: 
if, in a configuration $C$ where  $(b,v,\exclusive)\in\tau_i$
(resp.\ $(b,v,\shared)\in\tau_i)$,  a process $j\neq i$ applies any primitive 
(resp.\ any \emph{nontrivial} primitive) to $b$, then $\tau_i$ becomes
\emph{invalid} and any subsequent event invoked by $i$
sets $\tau_i$ to $\emptyset$ and returns $\bot$. We refer to this event as a \emph{tracking set abort}.

Note that hardware transactions may also abort spuriously, or because of unsupported operations~\cite{Rei12}, or due to \emph{capacity} aborts. 
Since modelling these aborts will not affect our results, except make for a more cumbersome presentation, we only consider capacity aborts in this paper. 

Any transaction $T_k \in \ms{txns}(E)$ that performs at least one cached access necessarily performs a \emph{cache-commit} primitive as the last event of $E|k$. 
A \emph{cache-commit} primitive issued by process $i$ with
a valid $\tau_i$ does the following: for each base object $b$ such that $(b,v,\exclusive) \in \tau_i$, the value of $b$ in $C$ is updated to $v$. 
Finally, $\tau_i$ is set to $\emptyset$ and the primitive 
returns $\textit{commit}$. 

\paragraph{Slow-path and fast-path transactions.}
In the following, we partition HyTM transactions into \emph{fast-path transactions} and \emph{slow-path transactions}.
Practically, two separate algorithms (fast-path one and slow-path one) 
are provided for each t-operation. 

A slow-path transaction models a regular software transaction.
An event of a slow-path transaction is either an invocation or response of a t-operation, or
a direct rmw primitive on a base object. 

A fast-path transaction essentially encapsulates a hardware transaction. Specifically, in any execution $E$,
we say that a transaction $T_k\in \ms{txns}(E)$ is a fast-path transaction if $E|k$ contains at least one cached event.
An event of a \emph{hardware transaction} includes series of direct trivial accesses and at least one cached access
followed by a \emph{cache-commit} primitive.
Note that we specifically allow hardware transactions to perform reads without adding the corresponding base object to
the process's tracking set, thus modelling the non-speculative accesses inside hardware allowed by 
IBM Power8 architectures. We remark that Intel Haswell does not support this feature: an event of a hardware transaction
does not include any direct access.

We assume that a fast-path transaction $T_k$ returns $A_k$
as soon an underlying cached primitive or \emph{cache-commit} returns $\bot$.
This implies the following lemma:
%
\begin{lemma}
\label{lm:traborts}
%
For any t-incomplete transaction $T_k \in \ms{txns}(E)$ executed by process $i$ and $(b,v,\exclusive)\in\tau_i$ (and resp. $(b,v,\shared)\in\tau_i$) after execution $E$
and let $e$ be any event (and resp. nontrivial event) that some process $j\neq i$ is poised to apply after $E$, then the next event of $T_k$ in any extension of $E\cdot e$
is $A_k$.
\end{lemma}
%
%

\paragraph{HyTM properties.}