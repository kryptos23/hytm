%!TEX root = htm.tex
\subsection{Instrumentation-optimal progressive HyTM }
\label{sec:hytm1}
%
%
For every t-object $X_j$, our implementation maintains a base object $v_j\in \mathbb{D}$ that stores the value of $X_j$
and a \emph{sequence lock} $r_{j}$. The sequence lock is an unsigned integer whose LSB bit stores the \emph{locked} state.
Specifically, we say that process $p_i$ \emph{holds a lock on $X_j$ after an execution $E$} if
$\textit{or}_j$ $\mathrel{\&} 1=1$ after $E$, where $\textit{or}_j$ is the value of $r_j$ after $E$.

\vspace{1mm}\noindent\textbf{Fast-path transactions.}
For a fast-path transaction $T_k$ executed by process $p_i$, the $\Read_k(X_j)$ implementation first reads $r_j$ (uncached)
and returns $A_k$ if some other process $p_j$ holds a lock on $X_j$.
Otherwise, it returns the value of $X_j$.
Updating fast-path transactions 
As with $\Read_k(X_j)$, the $\Write (X_j,v)$ implementation returns $A_k$ if some other process $p_j$ holds a lock on $X_j$.
Process $p_i$ then increments the value of $r_j$ by $2$ via a direct access and stores the cached state of $X_j$ along with its value $v$.
If the cache has not been invalidated, $p_i$ updates the shared memory
during $\TryC_k$ by invoking the $\ms{commit-cache}$ primitive.

\vspace{1mm}\noindent\textbf{Slow-path read-only transactions.}
Any $\Read_k(X_j)$ invoked by a slow-path transaction first reads the value of the object from $v_j$, 
checks if $r_j$ is se, adds $r_j$ to $\Rset(T_k)$
and then performs \emph{validation} on its entire read set to check if any of them have been modified. 
If either of these conditions is true,
the transaction returns $A_k$. Otherwise, it returns the value of $X_j$. 
Validation of the read set is performed by re-reading the values of the sequence lock entires stored in $\Rset(T_k)$.
A read-only transaction simply returns $C_k$ during the tryCommit.

\vspace{1mm}\noindent\textbf{Slow-path updating transactions.}
The $\Write_k(X,v)$ implementation of a slow-path transaction stores
$v$ and the current value of $X_j$ locally, 
deferring the actual update in shared memory to tryCommit. 
An updating slow-path transaction $T_k$ attempts to obtain exclusive write access to its 
entire write set by performing \emph{compare-and-set} (\emph{cas})
primitive that checks if the value of $r_j$, for each $X_j\in \Wset(T_k)$, is unchanged since last reading it during $\Write_k(X.v)$
If all the locks on the write set were acquired successfully, $T_k$ performs validation of the read set and returns $C_k$ if successful, else $p_i$ aborts the transaction.

\vspace{1mm}\noindent\textbf{Non-cached accesses inside fast-path.}

We can now prove the following theorem:
%
\begin{theorem}
\label{th:inswrite}
Algorithm~\ref{alg:inswrite} implements an opaque HyTM that provides invisible reads, progressiveness
such that in its every execution $E$, every fast-path transaction $T\in \ms{txns}(E)$
accesses $m=|\Dset(T_k)|$ distinct metadata base objects.
\end{theorem}
%
%
\begin{algorithm}[!ht]
\caption{Progressive fast-path and slow-path opaque HyTM implementation; code for transaction $T_k$}
\label{alg:inswrite}
\begin{algorithmic}[1]
  	\begin{multicols}{2}
  	{
  	\footnotesize
	\Part{Shared objects}{
		\State $v_j$, value of each t-object $X_j$ 
		\State $r_{j}$, a versioned lock for each t-object $X_j$
	}\EndPart	
	\Statex
	\Part{Process local objects}{
		\State $\Rset(T_k)$, storing $\{X_j,r_j\}$
		\State $\Wset(T_k)$, storing $\{X_j, v_j\}$
	}\EndPart
	\Statex
	\textbf{Code for fast-path transactions}
	\Statex
	\Part{$\textit{read}_k(X_j)$}\quad\Comment{fast-path}{
		\State $\textit{ov}_j := \Read(v_j)$ \Comment{cached read} \label{line:lin1}
		\State $\textit{or}_j := \Read(r_j)$ \Comment{uncached read}
		\If{$\textit{or}_j$ $\mathrel{\&}1$}  \label{line:hread}
			\Return $A_k$ \EndReturn
		\EndIf
		
		\Return $\textit{ov}_j$ \EndReturn
		
   	 }\EndPart
	\Statex
	%\Comment{What is the best strategy to buffer writes?}
	\Part{$\textit{write}_k(X_j,v)$}{\quad\Comment{fast-path}
		\State $\textit{or}_j := \Read(r_j)$ \Comment{cached read}
		\If{$\textit{or}_j$ $\mathrel{\&} 1$}  		
			\Return $A_k$ \EndReturn
		\EndIf
		
		\State $\Write(r_j,\textit{or}_j+2)$ \Comment{uncached write}
		\State $\Write(v_j,v)$ \Comment{uncached write} 
		\Return $\ok$ \EndReturn
		
   	}\EndPart
	\Statex
	
	\Part{$\textit{tryC}_k$()}{\quad\Comment{fast-path}
		\State $\ms{commit-cache}_i$ \label{line:lin3} \Comment{returns $C_k$ or $A_k$}
  	 }\EndPart
  	 
  	 \Statex
  	\Part{Function: $\lit{release}(Q)$}{
  		\ForAll{$X_j \in Q$}	
 			\State $r_j.\lit{write}(or_j+1)$ \label{line:rel1}	
		\EndFor
		
	}\EndPart
 	\Statex
	\Part{Function: $\lit{acquire}(Q)$}{
  		\ForAll{$X_j \in Q$}	
 			\If{ $r_j.\lit{setV}()$} \label{line:acq1}
			  \State $\ms{Lset}(T_k):=\ms{Lset}(T_k)\cup \{X_j\}$
			  \Return $\true$ \EndReturn
			\EndIf
			\State $\lit{release}(\ms{Lset}(T_k))$
			\Return $\false$ \EndReturn
		\EndFor
		
	}\EndPart
	\Statex
	\Statex \Comment{Implement using LL/SC on Power8}
	\Part{Function: $\lit{setV}()$}{
% 		\State $\ms{success} \gets \lit{false}$
% 		
% 		\While{($\neg \ms{success}$)} \Comment {spin until we get the lock}
		%\State $\ms{or}_j \gets$ $r_j.\Read()$ $\mathrel{\&}1111...1110$
		\If{$r_j$.CAS($or_j$, $or_j+1$)} 
		  \Return $\false$  \EndReturn
		\EndIf
		\Return $\true$ \EndReturn
	}\EndPart
  	 
  	 \newpage
	\textbf{Code for slow-path transactions}
	\Statex
	\Part{\Read$_k(X_j)$}\quad\Comment{slow-path}{
		  \If{$X_j\in \Wset(T_k)$}
		    \Return $\Wset(T_k).\lit{locate}(X_j)$ \EndReturn
		  \Else
		  
		  \State $\textit{or}_j := \Read(r_j)$ \label{line:readorec}
		  \State $\textit{ov}_j := \Read(v_j)$ \label{line:read2}
		  \State $\Rset(T_k) := \Rset(T_k)\cup\{X_j,or_j\}$ \label{line:rset}
		  \If{$\textit{or}_j$ $\mathrel{\&} 1$} \label{line:abort0}	
			\Return $A_k$ \EndReturn
		  \EndIf
		 
		  \If{$\neg \lit{validate}()$} \label{line:valid}
			\Return $A_k$ \EndReturn
		  \EndIf
		  \EndIf
		  \Return $\textit{ov}_j$ \EndReturn
		
   	 }\EndPart
	\Statex
	\Part{\Write$_k(X_j,v)$}\quad\Comment{slow-path}{
		
			\State $\textit{or}_j := \Read(r_j)$
			\State $\textit{nv}_j := v$
			\If{$\textit{or}_j$ $\mathrel{\&} 1$}	
			\Return $A_k$ \EndReturn
			\EndIf
			\State $\Wset(T_k) := \Wset(T_k)\cup\{X_j,\textit{nv}_j\}$
			\Return $\ok$ \EndReturn
		
   	}\EndPart
	\Statex
	
	%\Statex	
	\Part{\TryC$_k$()}\quad\Comment{slow-path}{
		\If{$\Wset(T_k)= \emptyset$}
			\Return $C_k$ \EndReturn \label{line:return}
		\EndIf
		\If{$\lit{acquire}(\Wset(T_k))$}	\label{line:acq}
		
		\If{$\neq \lit{validate}()$} \label{line:abort3}
			\State $\lit{release}( \ms{Wset}(T_k))$ 
			\Return $A_k$ \EndReturn
		\EndIf
		\ForAll{$X_j \in \Wset(T_k)$}
	 		\State  $v_j.\lit{write}(\textit{nv}_j)$ \label{line:write}
			 
	 	\EndFor	
		  
		\State $\lit{release}(\Wset(T_k))$   \label{line:rel}	
   		\Return $C_k$ \EndReturn
   		\Else
		  \Return $A_k$ \EndReturn
		  \EndIf
   	 }\EndPart		
	 
 	
	\Statex
	\Part{Function: $\lit{validate}()$}{\quad\Comment{Validate slow-path reading transactions}
		\If{$\exists X_j \in Rset(T_k)$;$X_j \not\in \Wset(T_k)$:$(\textit{or}_j\neq \Read(r_j))$} \label{line:valid}
			\Return $\false$ \EndReturn
		  \EndIf
		 \Return $\true$ \EndReturn
	}\EndPart
		
  	 }
	\end{multicols}
  \end{algorithmic}
\end{algorithm}
%
%%%%%%%%%%%%%%%%%%%%%%%%%%%%%%%%%%%%%%%%%%%%%%%%
\subsection{Instrumentation-optimal HyTMs that are progressive only for a subset of transactions}
\label{sec:hytm2}
%
In this section, we describe Algorithm~\ref{alg:inswrite2} which does not incur the linear instrumentation cost
on the fast-path reading transactions (as in Algorithm~\ref{alg:inswrite}, but provides progressiveness only
for slow-path reading transactions.
%
\begin{theorem}
\label{th:inswrite2}
Algorithm~\ref{alg:inswrite2} implements an opaque HyTM that provides invisible reads
such that (1) in every execution $E$,
every fast-path transaction $T\in \ms{txns}(E)$
accesses one metadata base object,
(2) every execution $E$ is fast-path progressive or for
every fast-path transaction $T\in \ms{txns}(E)$
that returns $A_k$ in $E$, there exists an updating slow-path transaction $T_m \in \ms{txns}(E)$
concurrent to $T_k$.
\end{theorem}
%
%%%%%%%%%%%%%%%%%%%%%%%%%%%%%%%%%%%%%%%%%%%%%%%
%
\begin{algorithm*}[!ht]
\caption{Opaque HyTM implementation with sequential slow-path and progressive fast-path TM-progress; code for $T_k$ by process $p_i$}
\label{alg:inswrite2}
\begin{algorithmic}[1]
  	\begin{multicols}{2}
  	{
  	\footnotesize
	\Part{Shared objects}{
		\State $v_j \in \mathbb{D}$, for each t-object $X_j$ 
		%\Statex ~~~~~allows reads, writes
		\State $r_{j}$, sequence lock for each t-object $X_j$
		\State $L$, global single-bit lock
		%\Statex ~~~~~allows reads, writes 
		%\State implemented from reads and writes
		%\State $L$, multi-trylock
	}\EndPart	
% 	\Statex
% 	\Part{Process local objects}{
% 		\State $\Rset(T_k)$, storing $\{X_j,r_j\}$
% 		\State $\Wset(T_k)$, storing $\{X_j, v_j\}$
% 	}\EndPart
	\Statex
	\textbf{Code for fast-path transactions}	
	\Statex
	\Part{$\textit{start}_k()$}{
		
		\State $l \gets \Read(\ms{L})$ \Comment{cached read} 
		\If{$\ms{l} \mathrel{\&} 1\neq 0$}
			    \Return $A_k$ \EndReturn
		\EndIf
		
	}\EndPart
	\Statex
	%\Comment{In general, would it better to buffer writes in tryC?}
	\Part{$\textit{read}_k(X_j)$}{\quad\Comment{fast-path}
		
		\State $\textit{ov}_j := v_j.\Read()$ \Comment{cached read} 
		
		\Return $\textit{ov}_j$ \EndReturn
		
   	 }\EndPart
	\Statex
	\Part{$\textit{write}_k(X_j,v)$}{\quad\Comment{fast-path}
	
	\State $\underline{\textit{or}_j := \Read(r_j)}$ \Comment{uncached read}
	\State $\underline{r_j.\Write({or}_j+2)}$ \Comment{uncached write}
	\State $v_j.\Write(v)$ \Comment{cached write}
	\Return $\ok$ \EndReturn
		
   	}\EndPart
	\Statex
	
	%\Statex	
	\Part{$\textit{tryC}_k$()}{\quad\Comment{fast-path}
		%\Return $C_k$ \EndReturn
		\State $\ms{commit-cache}_i$ \Comment{returns $C_k$}
		
   		
   	 }\EndPart		
   	 \Statex
	\Statex
	\textbf{Code for slow-path transactions}
	\Statex
	\Part{$\textit{start}_k()$}{
		
		\State $l \gets \Read(\ms{L})$
				
	}\EndPart
	\Statex
	\Part{\Read$_k(X_j)$}\quad\Comment{slow-path}{
		  \If{$X_j\in \Wset(T_k)$}
		    \Return $\Wset(T_k).\lit{locate}(X_j)$ \EndReturn
		  \Else
		  \State $\textit{ov}_j := \Read(v_j)$ 
		  \State $\textit{or}_j := \Read(r_j)$ 
		  \State $\Rset(T_k) := \Rset(T_k)\cup\{X_j,or_j\}$ 
		  \If{$\textit{or}_j$ $\mathrel{\&} 1$}  	
			\Return $A_k$ \EndReturn
		  \EndIf
		  \If{$\neg \lit{validate}()$}
			\Return $A_k$ \EndReturn
		  \EndIf
		  \EndIf
		  \Return $\textit{ov}_j$ \EndReturn
		
   	 }\EndPart
   	\newpage
	\Part{\Write$_k(X_j,v)$}\quad\Comment{slow-path}{
		
			\State $\textit{nv}_j := v$
			\State $\Wset(T_k) := \Wset(T_k)\cup\{X_j,nv_j\}$
			\Return $\ok$ \EndReturn
		
   	}\EndPart
	\Statex
	
	%\Statex	
	\Part{\TryC$_k$()}\quad\Comment{slow-path}{
		\If{$\Wset(T_k)= \emptyset$}
			\Return $C_k$ \EndReturn 
		\EndIf
				
		\While {$\neg flag$}
		  \State $flag \gets L.\lit{cas}(0,1)$
		\EndWhile
		\ForAll{$X_j \in Q$}	
			\State $or_j\gets r_j.\lit{read}()$ 
			 		
		\EndFor
		\Comment{First read, then write all: single barrier}
		\ForAll{$X_j \in Q$}	
			\State $r_j.\lit{write}(or_j + 1)$ 
			 		
		\EndFor
		\If{$\lit{validate}()$}
			\State \textbf{goto} Line~\ref{line:release}
			
		\EndIf
		
		\ForAll{$X_j \in \Wset(T_k)$}
	 		 \State  $v_j.\lit{write}(\textit{nv}_j)$
			 
			 
	 	\EndFor		
		
  		\ForAll{$X_j \in \Wset(T_k)$}	\label{line:release}
 			\State $r_j.\lit{write}(or_j + 1)$ 
		\EndFor
		
		\State $L.\lit{write}(1)$
   		\Return $C_k$ \EndReturn
   	 }\EndPart		
	\Statex
	\Part{Function: $\lit{validate}()$}{
		
		\If{$\exists X_j \in Rset(T_k)$:$(\textit{or}_j\neq \Read(r_j))$}
			\Return $\false$ \EndReturn
		  \EndIf
		
		 \Return $\true$ \EndReturn
	}\EndPart
	
% 	
	}
	\end{multicols}
  \end{algorithmic}
\end{algorithm*}

%%%%%%%%%%%%%%%%%%%%%%%%%%%%
\subsection{Correctness proofs}
\label{sec:proofs}
%
%!TEX root = htm.tex
\section{Correctness proofs}
\label{app:proofs}
%
We will prove the opacity of Algorithm~\ref{alg:inswrite} even if some of accesses performed by fast-path transactions are direct (as indicated in the pseudocode).
Analogous arguments apply to Algorithm~\ref{alg:inswrite2}.
%
\begin{lemma}
\label{lm:opacityh1}
Algorithm~\ref{alg:inswrite} implements an opaque TM.
\end{lemma}
%
\begin{proof}
%
Let $E$ by any execution of Algorithm~\ref{alg:inswrite}. 
Since opacity is a safety property, it is sufficient to prove that every finite execution is opaque~\cite{icdcs-opacity}.
Let $<_E$ denote a total-order on events in $E$.

Let $H$ denote a subsequence of $E$ constructed by selecting
\emph{linearization points} of t-operations performed in $E$.
The linearization point of a t-operation $op$, denoted as $\ell_{op}$ is associated with  
a base object event or an event performed during 
the execution of $op$ using the following procedure. 

\vspace{1mm}\noindent\textbf{Completions.}
First, we obtain a completion of $E$ by removing some pending
invocations or adding responses to the remaining pending invocations
as follows:
%
\begin{itemize}
\item
incomplete $\Read_k$, $\Write_k$ operation performed by a slow-path transaction $T_k$ is removed from $E$;
an incomplete $\TryC_k$ is removed from $E$ if $T_k$ has not performed any write to a base object $r_j$; $X_j \in \Wset(T_k)$
in Line~\ref{line:write}, otherwise it is completed by including $C_k$ after $E$.
\item
every incomplete $\Read_k$, $\TryA_k$, $\Write_k$ and $\TryC_k$ performed by a fast-path transaction $T_k$ is removed from $E$.
\end{itemize}
%
\vspace{1mm}\noindent\textbf{Linearization points.}
Now a linearization $H$ of $E$ is obtained by associating linearization points to
t-operations in the obtained completion of $E$.
For all t-operations performed a slow-path transaction $T_k$, linearization points as assigned as follows:
%
\begin{itemize}
\item For every t-read $op_k$ that returns a non-A$_k$ value, $\ell_{op_k}$ is chosen as the event in Line~\ref{line:read2}
of Algorithm~\ref{alg:inswrite}, else, $\ell_{op_k}$ is chosen as invocation event of $op_k$
\item For every $op_k=\Write_k$ that returns, $\ell_{op_k}$ is chosen as the invocation event of $op_k$
\item For every $op_k=\TryC_k$ that returns $C_k$ such that $\Wset(T_k)
  \neq \emptyset$, $\ell_{op_k}$ is associated with the first write to a base object performed by $\lit{release}$
  when invoked in Line~\ref{line:rel}, 
  else if $op_k$ returns $A_k$, $\ell_{op_k}$ is associated with the invocation event of $op_k$
\item For every $op_k=\TryC_k$ that returns $C_k$ such that $\Wset(T_k) = \emptyset$, 
$\ell_{op_k}$ is associated with Line~\ref{line:return}
\end{itemize}
%
For all t-operations performed a fast-path transaction $T_k$, linearization points as assigned as follows:
\begin{itemize}
\item For every t-read $op_k$ that returns a non-A$_k$ value, $\ell_{op_k}$ is chosen as the event in Line~\ref{line:lin1}
of Algorithm~\ref{alg:inswrite}, else, $\ell_{op_k}$ is chosen as invocation event of $op_k$
\item
For every $op_k$ that is a $\TryC_k$, $\ell_{op_k}$ is the $\ms{commit-cache}_k$ primitive invoked by $T_k$
\item
For every $op_k$ that is a $\Write_k$, $\ell_{op_k}$ is the event in Line~\ref{line:lin2}.
\end{itemize}
%
$<_H$ denotes a total-order on t-operations in the complete sequential history $H$.

\vspace{1mm}\noindent\textbf{Serialization points.}
The serialization of a transaction $T_j$, denoted as $\delta_{T_j}$ is
associated with the linearization point of a t-operation 
performed by the transaction.

We obtain a t-complete history ${\bar H}$ from $H$ as follows. 
A serialization $S$ is obtained by associating serialization points to transactions in ${\bar H}$ as follows:
for every transaction $T_k$ in $H$ that is complete, but not t-complete, 
we insert $\textit{tryC}_k\cdot A_k$ immediately 
after the last event of $T_k$ in $H$. 
%
\begin{itemize}
\item If $T_k$ is an updating transaction that commits, then $\delta_{T_k}$ is $\ell_{\TryC_k}$
\item If $T_k$ is a read-only or aborted transaction,
then $\delta_{T_k}$ is assigned to the linearization point of the last t-read that returned a non-A$_k$ value in $T_k$
\end{itemize}
%
$<_S$ denotes a total-order on transactions in the t-sequential history $S$.
Since for a given transaction, its
serialization point is chosen between the first and last event of the transaction,
if $T_i \prec_{H} T_j$, then $\delta_{T_i} <_{E} \delta_{T_j}$ implies $T_i <_S T_j$.
%
\begin{claim}
\label{cl:alg1claim}
%
If process $p_i$ executing transaction $T_k\in \ms{txns}(E)$ holds the lock on $X_j\in \ms{Wset}(T_k)$ after $E$, then the value $r_j$ is an odd integer value.
\end{claim}
%
\begin{proof}
%
\end{proof}
%
\begin{claim}
\label{cl:readfrom}
$S$ is legal.
\end{claim}
%
\begin{proof}
%
We claim that for every $\Read_j(X_m) \rightarrow v$, there exists some slow-path transaction $T_i$ (or resp. fast-path)
that performs $\Write_i(X_m,v)$ and completes the event in Line~\ref{line:write} (or resp. Line~\ref{line:lin2}) such that
$\Read_j(X_m) \not\prec_H^{RT} \Write_i(X_m,v)$.

Suppose that $T_i$ is a slow-path transaction:
since $\Read_j(X_m)$ returns the response $v$, the event in Line~\ref{line:read2}
succeeds the event in Line~\ref{line:write} performed by $\TryC_i$. 
Since $\Read_j(X_m)$ can return a non-abort response only after $T_i$ writes $0$ to $r_m$ in
Line~\ref{line:rel1}, $T_i$ must be committed in $S$.
Consequently,
$\ell_{\TryC_i} <_E \ell_{\Read_j(X_m)}$.
Since, for any updating
committing transaction $T_i$, $\delta_{T_i}=\ell_{\TryC_i}$, it follows that
$\delta_{T_{i}} <_E \delta_{T_{j}}$.

Otherwise if $T_i$ is a fast-path transaction, then clearly $T_i$ is a committed transaction in $S$.
Recall that $\Read_j(X_m)$ can read $v$ during the event in Line~\ref{line:read2}
only after $T_i$ applies the $\ms{commit-cache}$ primitive.
By the assignment of linearization points, 
$\ell_{\TryC_i} <_E \ell_{\Read_j(X_m)}$ and thus, $\delta_{T_{i}} <_E \ell_{\Read_j(X_m)}$.

Thus, to prove that $S$ is legal, it suffices to show that  
there does not exist a
transaction $T_k$ that returns $C_k$ in $S$ and performs $\Write_k(X_m,v')$; $v'\neq v$ such that $T_i <_S T_k <_S T_j$. 
%

$T_i$ and $T_k$ are both updating transactions that commit. Thus, 
%
\begin{center}
($T_i <_S T_k$) $\Longleftrightarrow$ ($\delta_{T_i} <_{E} \delta_{T_k}$) \\
($\delta_{T_i} <_{E} \delta_{T_k}$) $\Longleftrightarrow$ ($\ell_{\TryC_i} <_{E} \ell_{\TryC_k}$) 
\end{center}
%
Since, $T_j$ reads the value of $X$ written by $T_i$, one of the following is true:
$\ell_{\TryC_i} <_{E} \ell_{\TryC_k} <_{E} \ell_{\Read_j(X_m)}$ or
$\ell_{\TryC_i} <_{E} \ell_{\Read_j(X_m)} <_{E} \ell_{\TryC_k}$.

Suppose that $\ell_{\TryC_i} <_{E} \ell_{\TryC_k} <_{E} \ell_{\Read_j(X_m)}$.

(\textit{Case \RNum{1}:}) $T_i$ and $T_k$ are slow-path transactions.

Thus, $T_k$ returns a response from the event in Line~\ref{line:acq} 
before the read of the base object associated with $X_m$ by $T_j$ in Line~\ref{line:read2}. 
Since $T_i$ and $T_k$ are both committed in $E$, $T_k$ returns \emph{true} from the event in
Line~\ref{line:acq} only after $T_i$ writes $0$ to $r_{m}$ in Line~\ref{line:rel1}.

If $T_j$ is a slow-path transaction, 
recall that $\Read_j(X_m)$ checks if $X_j$ is locked by a concurrent transaction, 
then performs read-validation (Line~\ref{line:abort0}) before returning a matching response. 
We claim that $\Read_j(X_m)$ must return $A_j$ in any such execution.

Consider the following possible sequence of events: 
$T_k$ returns \emph{true} from \emph{acquire} function invocation, 
updates the value of $X_m$ to shared-memory (Line~\ref{line:write}), 
$T_j$ reads the base object $v_m$ associated with $X_m$, 
$T_k$ releases $X_m$ by writing $0$ to $r_{m}$ and finally $T_j$ performs the check in Line~\ref{line:abort0}. 
But in this case, $\Read_j(X_m)$ is forced to return the value $v'$ written by $T_m$--- 
contradiction to the assumption that $\Read_j(X_m)$ returns $v$. 

Otherwise suppose that $T_k$ acquires exclusive access to $X_m$ by writing $1$ to $r_{m}$ and returns \emph{true}
from the invocation of \emph{acquire}, updates $v_m$ in Line~\ref{line:write}), 
$T_j$ reads $v_m$, $T_j$ performs the check in Line~\ref{line:abort0} and finally $T_k$ 
releases $X_m$ by writing $0$ to $r_{m}$. 
Again, $\Read_j(X_m)$ must return $A_j$ since $T_j$ reads that $r_{m}$ is $1$---contradiction.

A similar argument applies to the case that $T_j$ is a fast-path transaction.
Indeed, since every \emph{data} base object read by $T_j$ is contained in its tracking set, if any concurrent
transaction updates any t-object in its read set, $T_j$ is aborted immediately by our model(cf. Section~\ref{sec:hytm}).

Thus, $\ell_{\TryC_i} <_E \ell_{\Read_j(X)} <_{E} \ell_{\TryC_k}$.

(\textit{Case \RNum{2}:}) $T_i$ is a slow-path transaction and $T_k$ is a fast-path transaction.
Thus, $T_k$ returns $C_k$ 
before the read of the base object associated with $X_m$ by $T_j$ in Line~\ref{line:read2}, but after the response
of \emph{acquire} by $T_i$ in Line~\ref{line:acq}.
Since $\Read_j(X_m)$ reads the value of $X_m$ to be $v$ and not $v'$, $T_i$ performs the \emph{cas}
to $v_m$ in Line~\ref{line:write} after the $T_k$ performs the $\ms{commit-cache}$ primitive (since if
otherwise, $T_k$ would be aborted in $E$).
But then the \emph{cas} on $v_m$ performed by $T_i$ would return $\false$ and $T_i$ would return $A_i$---contradiction.

(\textit{Case \RNum{3}:}) $T_k$ is a slow-path transaction and $T_i$ is a fast-path transaction.
This is analogous to the above case.

(\textit{Case \RNum{4}:}) $T_i$ and $T_k$ are fast-path transactions.
Thus, $T_k$ returns $C_k$ 
before the read of the base object associated with $X_m$ by $T_j$ in Line~\ref{line:read2}, but before $T_i$
returns $C_i$ (this follows from Observation).
Consequently, $\Read_j(X_m)$ must read the value of $X_m$ to be $v'$ and return $v'$---contradiction.
%

We now need to prove that $\delta_{T_{j}}$ indeed precedes $\ell_{\TryC_k}$ in $E$.

Consider the two possible cases:
%
\begin{itemize}
\item
Suppose that $T_j$ is a read-only transaction. 
Then, $\delta_{T_j}$ is assigned to the last t-read performed by $T_j$ that returns a non-A$_j$ value. 
If $\Read_j(X_m)$ is not the last t-read that returned a non-A$_j$ value, then there exists a $\Read_j(X')$ such that 
$\ell_{\Read_j(X_m)} <_{E} \ell_{\TryC_k} <_E \ell_{read_j(X')}$.
But then this t-read of $X'$ must abort by performing the checks in Line~\ref{line:abort0} or incur a tracking set abort---contradiction.
\item
Suppose that $T_j$ is an updating transaction that commits, then $\delta_{T_j}=\ell_{\TryC_j}$ which implies that
$\ell_{read_j(X)} <_{E} \ell_{\TryC_k} <_E \ell_{\TryC_j}$. Then, $T_j$ must neccesarily perform the checks
in Line~\ref{line:abort3} and return $A_j$ or incur a tracking set abort---contradiction to the assumption that $T_j$ is a committed transaction.
\end{itemize}
%
The proof follows.
%
\end{proof}
%
Since $S$ is legal and respects the real-time ordering of transactions, Algorithm~\ref{alg:inswrite} is opaque.
%
\end{proof}
%

%%%%%%%%%%%%%%%%%%%%%%%%%%%%%%%%
\subsection{Minimizing the cost for incremental validation in opaque HyTMs}
\label{sec:middlepath}
%
Observe that the lower bound in Theorem~\ref{th:impossibility} assumes progressiveness for both slow-path and fast-path transactions
along with opacity and invisible reads.
In this section, we present algorithmic ideas for cirvumventing the lower bound or minimizing the cost incurred
by implementations due to incremental validation.

\paragraph{Sacrificing progressiveness and minimizing contention window.}
%
\emph{Hybrid Norec}~\cite{hybridnorec} is a HyTM implementation that does not satisfy progressiveness
(unlike its STM counterpart Norec), but mitigates
the step-complexity cost on slow-path transactions by performing incremental validation during a transactional read \emph{iff} 
the global snapshot of the memory locations has changed since the start of the transaction.
%
%\begin{algorithm*}[!h]
\caption{HybridNorec HyTM implementation; code for $T_k$ by process $p_i$}
\label{alg:inswrite3}
\begin{algorithmic}[1]
  	\begin{multicols}{2}
  	{
  	\footnotesize
	\Part{Shared objects}{
		\State $\ms{gsl}$, global sequence lock 
		%\Statex ~~~~~allows reads, writes
		\State $\ms{esl}$, extra sequence lock-bit
		
	}\EndPart	
 	\Statex
 	\Part{Process local objects}{
 		\State $\ms{sl}_i$, process $p_i$'s local sequence lock
 		
 	}\EndPart
	\Statex
	\textbf{Code for fast-path transactions}	
	\Statex
	\Part{$\textit{start}_k()$}{
%		\State $ \ms{l}   \gets \Read(\ms{esl})$
		\If {$\Read(\ms{esl})$}
		  \Return $A_k$ \EndReturn
		\EndIf
		
	}\EndPart
	\Statex
	%\Comment{In general, would it better to buffer writes in tryC?}
	\Part{$\textit{read}_k(X_j)$}{\quad\Comment{fast-path}
		
		\State $\textit{ov}_j := v_j.\Read()$ \Comment{cached read} 
		
		\Return $\textit{ov}_j$ \EndReturn
		
   	 }\EndPart
	\Statex
	\Part{$\textit{write}_k(X_j,v)$}{\quad\Comment{fast-path}
	
	\State $v_j.\Write(v)$ \Comment{cached write}
	\Return $\ok$ \EndReturn
		
   	}\EndPart
	\Statex
	
	%\Statex	
	\Part{$\textit{tryC}_k$()}{\quad\Comment{fast-path}
		%\Return $C_k$ \EndReturn
		\If{$\Wset(T_k)= \emptyset$}
			\Return $C_k$ \EndReturn 
		\EndIf
		\State $ \ms{l}   \gets \Read(\ms{gsl})$
		\If {$\ms{l} \mathrel{\&} 1$}
		  \Return $A_k$ \EndReturn
		\EndIf
		\State $\ms{gsl}.\lit{write}(l+2)$
		\State $\ms{commit-cache}_i$ \Comment{returns $C_k$}
		
   		
   	 }\EndPart		
   	 \Statex
	\Statex
	\textbf{Code for slow-path transactions}
	\Statex
	\Part{$\textit{start}_k()$}{
		\While {$\ms{sl} \mathrel{\&} 1$}
		  \State $ \ms{sl}   \gets \Read(\ms{gsl})$
		\EndWhile
		
	}\EndPart
	\Statex
	\Part{\Read$_k(X_j)$}\quad\Comment{slow-path}{
		  \If{$X_j\in \Wset(T_k)$}
		    \Return $\Wset(T_k).\lit{locate}(X_j)$ \EndReturn
		  \EndIf
		  \State $\textit{ov}_j := \Read(v_j)$ 
		  
		  \State $\Rset(T_k) := \Rset(T_k)\cup\{X_j,or_j\}$
		  \If{$\neg \lit{validate}()$}
			
			\Return $A_k$ \EndReturn
		  \EndIf
		  \Return $\textit{ov}_j$ \EndReturn
	
   	 }\EndPart
   	\newpage
	\Part{\Write$_k(X_j,v)$}\quad\Comment{slow-path}{
		
			\State $\textit{nv}_j := v$
			\State $\Wset(T_k) := \Wset(T_k)\cup\{X_j,nv_j\}$
			\Return $\ok$ \EndReturn
		
   	}\EndPart
	\Statex
	
	%\Statex	
	\Part{\TryC$_k$()}\quad\Comment{slow-path}{
		\If{$\Wset(T_k)= \emptyset$}
			\Return $C_k$ \EndReturn 
		\EndIf
		
		\While{$\ms{sl} \mathrel{\&} 1 \mbox{ \textbf{or} } \neg\ms{gsl}$.CAS($sl$, $sl+1$)} \Comment{Lock gsl} 
          \State $\ms{sl} \gets \Read(\ms{gsl})$
		\EndWhile
		\State $\ms{esl}.\lit{write}(1)$
		\If{$\neg\lit{validate}()$}
    		\State $\ms{esl}.\lit{write}(0)$
			\State $\ms{gsl}.\lit{write}(\lit{read}(\ms{gsl})+1)$
			\Return $A_k$ \EndReturn
		\EndIf
		\ForAll{$X_j \in \Wset(T_k)$}
	 		 \State  $v_j.\lit{write}(\textit{nv}_j)$
		\EndFor
		\State $\ms{esl}.\lit{write}(0)$
		\State $\ms{gsl}.\lit{write}(\lit{read}(\ms{gsl})+1)$
  		\Return $C_k$ \EndReturn
   	 }\EndPart		
	\Statex
	\Part{Function: $\lit{validate}()$}{
        \State $\ms{currGSL} \gets \Read(\ms{gsl})$
		\If{$\ms{sl} = \ms{currGSL}$}
            \Return $\true$ \EndReturn
        \EndIf
        \While{$true$}
            \While{$\ms{currGSL} \mathrel{\&} 1$}
                \State $\ms{currGSL} \gets \Read(\ms{gsl})$
            \EndWhile
    		\If{$\exists X_j \in Rset(T_k)$:$(\textit{ov}_j \neq \Read(X_j))$}
    			\Return $\false$ \EndReturn
    		\EndIf
            \If{$\Read(gsl) = \ms{currGSL}$}
                \State $\ms{sl} \gets \ms{currGSL}$
        		\Return $\true$ \EndReturn
            \EndIf
        \EndWhile
	}\EndPart
	
% 	
	}
	\end{multicols}
  \end{algorithmic}
\end{algorithm*}


\paragraph{Employing non-cached writes inside fast-path.}

\paragraph{Performing short fast-path transactions inside slow-path.}

\paragraph{Employing an uninstrumented fast fast-path.}
Note that ideally we would like to execute all transactions inside hardware with minimal instrumentation.
We now describe how every transaction may first be executed in a ``fast'' fast-path with almost no instrumentation
and if unsuccessful, may be re-attempted in the fast-path and subsequently in slow-path.
Specifically, Algorithm~\ref{alg:middle} describes a generic transformation for any opaque HyTM $\mathcal{M}$ to an opaque
HyTM $\mathcal{M}'$ by employing a shared \emph{fetch-and-add} metadata $F$ that slow-path updating transactions
increment (and resp. decrement) at the start (and resp. end). The fast fast-path checks first checks if $F$ is $0$
and if not, aborts the transaction; otherwise the transaction is continued as an uninstrumented hardware transaction.
Since there is no concurrency between fast fast-path and slow-path, the following result is immediate:
%
\begin{algorithm*}[!h]
\caption*{\textbf{Algorithm 1+} Transformation for opaque HyTM $\mathcal{M}$ to include a fast fast-path; code for $T_k$ by process $p_i$}
\label{alg:middle}
\vspace{-1mm}
\noindent\lstset{style=customc}
\begin{minipage}{0.49\textwidth}
\begin{lstlisting}[frame=none,firstnumber=1,mathescape=true]
//\textbf{Shared objects}
    F, fetch-and-increment object
    
//\textbf{Code for slow-path transactions}
tryC$_k$()
    if $\Wset(T_k)= \emptyset$ then
	return C$_k$  
    $F.\lit{fetch-add}(1)$
    //\textbf{Invoke updating slow-path tryC$_k$(); let $r_k$ be the response}
    $F.\lit{fetch-add}(-1)$
    return r$_k$ 
\end{lstlisting}
\end{minipage}
\begin{minipage}{0.49\textwidth}
\begin{lstlisting}[frame=none,firstnumber=last,mathescape=true]
//\textbf{Code for fast fast-path transactions}
start$_k$()
    if F $\neq 0$ then //\medcom cached read
	return A$_k$

read$_k$(X$_j$)
    ov$_j$ := v$_j$ //\medcom cached read 
    return ov$_j$
		
write$_k$(X$_j$,v)
    v$_j$:=v  //\medcom cached write
    return ok

tryC$_k$()
    commit-cache$_i$ //\medcom returns C$_k$ or A$_k$
\end{lstlisting}
\end{minipage}
\end{algorithm*}



% \begin{algorithmic}[1]
%   	\begin{multicols}{2}
%   	{
%   	\footnotesize
% 	\Part{Shared objects}{
% 		%\State $v_j$, for each t-object $X_j$ 
% 		\State $F$, fetch-and-increment object
% 		
% 	}\EndPart	
%  	\Statex
%  	\Statex
%  	\textbf{Code for slow-path transactions}
% 	\Statex
% 	\Part{$\textit{tryC}_k$()}{\quad\Comment{slow-path}
% 		\If{$\Wset(T_k)= \emptyset$}
% 			\Return $C_k$ \EndReturn 
% 		\EndIf
% 		\State $F.\lit{fetch-add}(1)$
% 		\State \textbf{Invoke updating slow-path $\TryC_k()$; let $r_k$ be the response}
% 		
% 		\State $F.\lit{fetch-add}(-1)$
% 		\Return $r_k$ \EndReturn
%    		
%    	 }\EndPart		
% 	\newpage
% 	\textbf{Code for fast fast-path transactions}	
% 	\Statex
% 	\Part{$\textit{start}_k()$}{
% 		
% 		\If{$\Read(\ms{F}) \neq 0$} \Comment{cached read}
% 		  \Return $A_k$ \EndReturn
% 		\EndIf
% 		
% 	}\EndPart
% 	\Statex
% 	%\Comment{In general, would it better to buffer writes in tryC?}
% 	\Part{$\textit{read}_k(X_j)$}{\quad\Comment{fast fast-path}
% 		
% 		\State $\textit{ov}_j := v_j.\Read()$ \Comment{cached read} 
% 		
% 		\Return $\textit{ov}_j$ \EndReturn
% 		
%    	 }\EndPart
% 	\Statex
% 	\Part{$\textit{write}_k(X_j,v)$}{\quad\Comment{fast fast-path}
% 	
% 	\State $v_j.\Write(v)$ \Comment{cached write}
% 	\Return $\ok$ \EndReturn
% 		
%    	}\EndPart
% 	\Statex
% 	
% 	%\Statex	
% 	\Part{$\textit{tryC}_k$()}{\quad\Comment{fast fast-path}
% 		%\Return $C_k$ \EndReturn
% 		\State $\ms{commit-cache}_i$ \Comment{returns $C_k$ or $A_k$}
% 		
%    		
%    	 }\EndPart		
%    	   	
% 	
% 	}
% 	\end{multicols}
%   \end{algorithmic}
% \end{algorithm*}

%
\begin{theorem}
Given any opaque HyTM $\mathcal{M}$ with a fast-path and slow-path, Algorithm~\ref{alg:middle} implements an opaque HyTM $\mathcal{M}'$
with a fast fast-path, fast-path and a slow-path.
\end{theorem}
