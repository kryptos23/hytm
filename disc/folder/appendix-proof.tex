%!TEX root = htm.tex
%
%%%%%%%%%%%%%%%%%%%%%%%%%%%%%%%%%%%%%%%%%%%%%%%%
\begin{algorithm}[!ht]
\caption{Progressive fast-path and slow-path opaque HyTM implementation; code for transaction $T_k$}
\label{alg:inswrite}
\begin{algorithmic}[1]
  	\begin{multicols}{2}
  	{
  	\footnotesize
	\Part{Shared objects}{
		\State $v_j$, value of each t-object $X_j$ 
		\State $r_{j}$, a versioned lock for each t-object $X_j$
	}\EndPart	
	\Statex
	\Part{Process local objects}{
		\State $\Rset(T_k)$, storing $\{X_j,r_j\}$
		\State $\Wset(T_k)$, storing $\{X_j, v_j\}$
	}\EndPart
	\Statex
	\textbf{Code for fast-path transactions}
	\Statex
	\Part{$\textit{read}_k(X_j)$}\quad\Comment{fast-path}{
		\State $\textit{ov}_j := \Read(v_j)$ \Comment{cached read} \label{line:lin1}
		\State $\textit{or}_j := \Read(r_j)$ \Comment{uncached read}
		\If{$\textit{or}_j$ $\mathrel{\&}1$}  \label{line:hread}
			\Return $A_k$ \EndReturn
		\EndIf
		
		\Return $\textit{ov}_j$ \EndReturn
		
   	 }\EndPart
	\Statex
	%\Comment{What is the best strategy to buffer writes?}
	\Part{$\textit{write}_k(X_j,v)$}{\quad\Comment{fast-path}
		\State $\textit{or}_j := \Read(r_j)$ \Comment{cached read}
		\If{$\textit{or}_j$ $\mathrel{\&} 1$}  		
			\Return $A_k$ \EndReturn
		\EndIf
		
		\State $\Write(r_j,\textit{or}_j+2)$ \Comment{uncached write}
		\State $\Write(v_j,v)$ \Comment{uncached write} 
		\Return $\ok$ \EndReturn
		
   	}\EndPart
	\Statex
	
	\Part{$\textit{tryC}_k$()}{\quad\Comment{fast-path}
		\State $\ms{commit-cache}_i$ \label{line:lin3} \Comment{returns $C_k$ or $A_k$}
  	 }\EndPart
  	 
  	 \Statex
  	\Part{Function: $\lit{release}(Q)$}{
  		\ForAll{$X_j \in Q$}	
 			\State $r_j.\lit{write}(or_j+1)$ \label{line:rel1}	
		\EndFor
		
	}\EndPart
 	\Statex
	\Part{Function: $\lit{acquire}(Q)$}{
  		\ForAll{$X_j \in Q$}	
 			\If{ $r_j.\lit{setV}()$} \label{line:acq1}
			  \State $\ms{Lset}(T_k):=\ms{Lset}(T_k)\cup \{X_j\}$
			  \Return $\true$ \EndReturn
			\EndIf
			\State $\lit{release}(\ms{Lset}(T_k))$
			\Return $\false$ \EndReturn
		\EndFor
		
	}\EndPart
	\Statex
	\Statex \Comment{Implement using LL/SC on Power8}
	\Part{Function: $\lit{setV}()$}{
% 		\State $\ms{success} \gets \lit{false}$
% 		
% 		\While{($\neg \ms{success}$)} \Comment {spin until we get the lock}
		%\State $\ms{or}_j \gets$ $r_j.\Read()$ $\mathrel{\&}1111...1110$
		\If{$r_j$.CAS($or_j$, $or_j+1$)} 
		  \Return $\false$  \EndReturn
		\EndIf
		\Return $\true$ \EndReturn
	}\EndPart
  	 
  	 \newpage
	\textbf{Code for slow-path transactions}
	\Statex
	\Part{\Read$_k(X_j)$}\quad\Comment{slow-path}{
		  \If{$X_j\in \Wset(T_k)$}
		    \Return $\Wset(T_k).\lit{locate}(X_j)$ \EndReturn
		  \Else
		  
		  \State $\textit{or}_j := \Read(r_j)$ \label{line:readorec}
		  \State $\textit{ov}_j := \Read(v_j)$ \label{line:read2}
		  \State $\Rset(T_k) := \Rset(T_k)\cup\{X_j,or_j\}$ \label{line:rset}
		  \If{$\textit{or}_j$ $\mathrel{\&} 1$} \label{line:abort0}	
			\Return $A_k$ \EndReturn
		  \EndIf
		 
		  \If{$\neg \lit{validate}()$} \label{line:valid}
			\Return $A_k$ \EndReturn
		  \EndIf
		  \EndIf
		  \Return $\textit{ov}_j$ \EndReturn
		
   	 }\EndPart
	\Statex
	\Part{\Write$_k(X_j,v)$}\quad\Comment{slow-path}{
		
			\State $\textit{or}_j := \Read(r_j)$
			\State $\textit{nv}_j := v$
			\If{$\textit{or}_j$ $\mathrel{\&} 1$}	
			\Return $A_k$ \EndReturn
			\EndIf
			\State $\Wset(T_k) := \Wset(T_k)\cup\{X_j,\textit{nv}_j\}$
			\Return $\ok$ \EndReturn
		
   	}\EndPart
	\Statex
	
	%\Statex	
	\Part{\TryC$_k$()}\quad\Comment{slow-path}{
		\If{$\Wset(T_k)= \emptyset$}
			\Return $C_k$ \EndReturn \label{line:return}
		\EndIf
		\If{$\lit{acquire}(\Wset(T_k))$}	\label{line:acq}
		
		\If{$\neq \lit{validate}()$} \label{line:abort3}
			\State $\lit{release}( \ms{Wset}(T_k))$ 
			\Return $A_k$ \EndReturn
		\EndIf
		\ForAll{$X_j \in \Wset(T_k)$}
	 		\State  $v_j.\lit{write}(\textit{nv}_j)$ \label{line:write}
			 
	 	\EndFor	
		  
		\State $\lit{release}(\Wset(T_k))$   \label{line:rel}	
   		\Return $C_k$ \EndReturn
   		\Else
		  \Return $A_k$ \EndReturn
		  \EndIf
   	 }\EndPart		
	 
 	
	\Statex
	\Part{Function: $\lit{validate}()$}{\quad\Comment{Validate slow-path reading transactions}
		\If{$\exists X_j \in Rset(T_k)$;$X_j \not\in \Wset(T_k)$:$(\textit{or}_j\neq \Read(r_j))$} \label{line:valid}
			\Return $\false$ \EndReturn
		  \EndIf
		 \Return $\true$ \EndReturn
	}\EndPart
		
  	 }
	\end{multicols}
  \end{algorithmic}
\end{algorithm}
%%%%%%%%%%%%%%%%%%%%%%%%%%%%%%%%%%%%%%%%%%%%%%%%
%%%%%%%%%%%%%%%%%%%%%%%%%%%%%
\begin{algorithm*}[!ht]
\caption{Opaque HyTM implementation with sequential slow-path and progressive fast-path TM-progress; code for $T_k$ by process $p_i$}
\label{alg:inswrite2}
\begin{algorithmic}[1]
  	\begin{multicols}{2}
  	{
  	\footnotesize
	\Part{Shared objects}{
		\State $v_j \in \mathbb{D}$, for each t-object $X_j$ 
		%\Statex ~~~~~allows reads, writes
		\State $r_{j}$, sequence lock for each t-object $X_j$
		\State $L$, global single-bit lock
		%\Statex ~~~~~allows reads, writes 
		%\State implemented from reads and writes
		%\State $L$, multi-trylock
	}\EndPart	
% 	\Statex
% 	\Part{Process local objects}{
% 		\State $\Rset(T_k)$, storing $\{X_j,r_j\}$
% 		\State $\Wset(T_k)$, storing $\{X_j, v_j\}$
% 	}\EndPart
	\Statex
	\textbf{Code for fast-path transactions}	
	\Statex
	\Part{$\textit{start}_k()$}{
		
		\State $l \gets \Read(\ms{L})$ \Comment{cached read} 
		\If{$\ms{l} \mathrel{\&} 1\neq 0$}
			    \Return $A_k$ \EndReturn
		\EndIf
		
	}\EndPart
	\Statex
	%\Comment{In general, would it better to buffer writes in tryC?}
	\Part{$\textit{read}_k(X_j)$}{\quad\Comment{fast-path}
		
		\State $\textit{ov}_j := v_j.\Read()$ \Comment{cached read} 
		
		\Return $\textit{ov}_j$ \EndReturn
		
   	 }\EndPart
	\Statex
	\Part{$\textit{write}_k(X_j,v)$}{\quad\Comment{fast-path}
	
	\State $\underline{\textit{or}_j := \Read(r_j)}$ \Comment{uncached read}
	\State $\underline{r_j.\Write({or}_j+2)}$ \Comment{uncached write}
	\State $v_j.\Write(v)$ \Comment{cached write}
	\Return $\ok$ \EndReturn
		
   	}\EndPart
	\Statex
	
	%\Statex	
	\Part{$\textit{tryC}_k$()}{\quad\Comment{fast-path}
		%\Return $C_k$ \EndReturn
		\State $\ms{commit-cache}_i$ \Comment{returns $C_k$}
		
   		
   	 }\EndPart		
   	 \Statex
	\Statex
	\textbf{Code for slow-path transactions}
	\Statex
	\Part{$\textit{start}_k()$}{
		
		\State $l \gets \Read(\ms{L})$
				
	}\EndPart
	\Statex
	\Part{\Read$_k(X_j)$}\quad\Comment{slow-path}{
		  \If{$X_j\in \Wset(T_k)$}
		    \Return $\Wset(T_k).\lit{locate}(X_j)$ \EndReturn
		  \Else
		  \State $\textit{ov}_j := \Read(v_j)$ 
		  \State $\textit{or}_j := \Read(r_j)$ 
		  \State $\Rset(T_k) := \Rset(T_k)\cup\{X_j,or_j\}$ 
		  \If{$\textit{or}_j$ $\mathrel{\&} 1$}  	
			\Return $A_k$ \EndReturn
		  \EndIf
		  \If{$\neg \lit{validate}()$}
			\Return $A_k$ \EndReturn
		  \EndIf
		  \EndIf
		  \Return $\textit{ov}_j$ \EndReturn
		
   	 }\EndPart
   	\newpage
	\Part{\Write$_k(X_j,v)$}\quad\Comment{slow-path}{
		
			\State $\textit{nv}_j := v$
			\State $\Wset(T_k) := \Wset(T_k)\cup\{X_j,nv_j\}$
			\Return $\ok$ \EndReturn
		
   	}\EndPart
	\Statex
	
	%\Statex	
	\Part{\TryC$_k$()}\quad\Comment{slow-path}{
		\If{$\Wset(T_k)= \emptyset$}
			\Return $C_k$ \EndReturn 
		\EndIf
				
		\While {$\neg flag$}
		  \State $flag \gets L.\lit{cas}(0,1)$
		\EndWhile
		\ForAll{$X_j \in Q$}	
			\State $or_j\gets r_j.\lit{read}()$ 
			 		
		\EndFor
		\Comment{First read, then write all: single barrier}
		\ForAll{$X_j \in Q$}	
			\State $r_j.\lit{write}(or_j + 1)$ 
			 		
		\EndFor
		\If{$\lit{validate}()$}
			\State \textbf{goto} Line~\ref{line:release}
			
		\EndIf
		
		\ForAll{$X_j \in \Wset(T_k)$}
	 		 \State  $v_j.\lit{write}(\textit{nv}_j)$
			 
			 
	 	\EndFor		
		
  		\ForAll{$X_j \in \Wset(T_k)$}	\label{line:release}
 			\State $r_j.\lit{write}(or_j + 1)$ 
		\EndFor
		
		\State $L.\lit{write}(1)$
   		\Return $C_k$ \EndReturn
   	 }\EndPart		
	\Statex
	\Part{Function: $\lit{validate}()$}{
		
		\If{$\exists X_j \in Rset(T_k)$:$(\textit{or}_j\neq \Read(r_j))$}
			\Return $\false$ \EndReturn
		  \EndIf
		
		 \Return $\true$ \EndReturn
	}\EndPart
	
% 	
	}
	\end{multicols}
  \end{algorithmic}
\end{algorithm*}

%%%%%%%%%%%%%%%%%%%%%%%%%%%%%%%%%%%%%%%%%%%%%%%%
\section{Proof of opacity for algorithms}
\label{app:opacity}
We will prove the opacity of Algorithm~\ref{alg:inswrite} even if some of accesses performed by fast-path transactions are direct (as indicated in the pseudocode).
Analogous arguments apply to Algorithm~\ref{alg:inswrite2}.
%

Let $E$ by any execution of Algorithm~\ref{alg:inswrite}. 
Since opacity is a safety property, it is sufficient to prove that every finite execution is opaque~\cite{icdcs-opacity}.
Let $<_E$ denote a total-order on events in $E$.

Let $H$ denote a subsequence of $E$ constructed by selecting
\emph{linearization points} of t-operations performed in $E$.
The linearization point of a t-operation $op$, denoted as $\ell_{op}$ is associated with  
a base object event or an event performed during 
the execution of $op$ using the following procedure. 

\vspace{1mm}\noindent\textbf{Completions.}
First, we obtain a completion of $E$ by removing some pending
invocations or adding responses to the remaining pending invocations.
Incomplete $\Read_k$, $\Write_k$ operation performed by a slow-path transaction $T_k$ is removed from $E$;
an incomplete $\TryC_k$ is removed from $E$ if $T_k$ has not performed any write to a base object $r_j$; $X_j \in \Wset(T_k)$
in Line~\ref{line:write}, otherwise it is completed by including $C_k$ after $E$.
Every incomplete $\Read_k$, $\TryA_k$, $\Write_k$ and $\TryC_k$ performed by a fast-path transaction $T_k$ is removed from $E$.

\vspace{1mm}\noindent\textbf{Linearization points.}
Now a linearization $H$ of $E$ is obtained by associating linearization points to
t-operations in the obtained completion of $E$.
For all t-operations performed a slow-path transaction $T_k$, linearization points as assigned as follows:
%
\begin{itemize}
\item For every t-read $op_k$ that returns a non-A$_k$ value, $\ell_{op_k}$ is chosen as the event in Line~\ref{line:read2}
of Algorithm~\ref{alg:inswrite}, else, $\ell_{op_k}$ is chosen as invocation event of $op_k$
\item For every $op_k=\Write_k$ that returns, $\ell_{op_k}$ is chosen as the invocation event of $op_k$
\item For every $op_k=\TryC_k$ that returns $C_k$ such that $\Wset(T_k)
  \neq \emptyset$, $\ell_{op_k}$ is associated with the first write to a base object performed by $\lit{release}$
  when invoked in Line~\ref{line:rel}, 
  else if $op_k$ returns $A_k$, $\ell_{op_k}$ is associated with the invocation event of $op_k$
\item For every $op_k=\TryC_k$ that returns $C_k$ such that $\Wset(T_k) = \emptyset$, 
$\ell_{op_k}$ is associated with Line~\ref{line:return}
\end{itemize}
%
For all t-operations performed a fast-path transaction $T_k$, linearization points are assigned as follows:
\begin{itemize}
\item For every t-read $op_k$ that returns a non-A$_k$ value, $\ell_{op_k}$ is chosen as the event in Line~\ref{line:lin1}
of Algorithm~\ref{alg:inswrite}, else, $\ell_{op_k}$ is chosen as invocation event of $op_k$
\item
For every $op_k$ that is a $\TryC_k$, $\ell_{op_k}$ is the $\ms{commit-cache}_k$ primitive invoked by $T_k$
\item
For every $op_k$ that is a $\Write_k$, $\ell_{op_k}$ is the event in Line~\ref{line:lin2}.
\end{itemize}
%
$<_H$ denotes a total-order on t-operations in the complete sequential history $H$.

\vspace{1mm}\noindent\textbf{Serialization points.}
The serialization of a transaction $T_j$, denoted as $\delta_{T_j}$ is
associated with the linearization point of a t-operation 
performed by the transaction.

We obtain a t-complete history ${\bar H}$ from $H$ as follows. 
A serialization $S$ is obtained by associating serialization points to transactions in ${\bar H}$ as follows:
for every transaction $T_k$ in $H$ that is complete, but not t-complete, 
we insert $\textit{tryC}_k\cdot A_k$ immediately 
after the last event of $T_k$ in $H$. 
If $T_k$ is an updating transaction that commits, then $\delta_{T_k}$ is $\ell_{\TryC_k}$.
If $T_k$ is a read-only or aborted transaction,
then $\delta_{T_k}$ is assigned to the linearization point of the last t-read that returned a non-A$_k$ value in $T_k$.

$<_S$ denotes a total-order on transactions in the t-sequential history $S$.
Since for a given transaction, its
serialization point is chosen between the first and last event of the transaction,
if $T_i \prec_{H} T_j$, then $\delta_{T_i} <_{E} \delta_{T_j}$ implies $T_i <_S T_j$.

Throughout this proof, we consider that process $p_i$ executing fast-path transaction $T_k \in \ms{txns}(E)$
does not include the sequence lock $r_j$ in the tracking set of $p_i$ when accessed in Line~\ref{line:hread}
during $\Read_k(X_j)$.
%
%
\begin{claim}
\label{cl:fast}
If every transaction $T_k \in \ms{txns}(E)$ is fast-path, then $S$ is legal.
\end{claim}
%
\begin{proof}
%
Recall that Algorithm~\ref{alg:inswrite} performs direct accesses only during the t-read operation in Line~\ref{line:hread} which involves reading the sequence lock $r_j$ corresponding to t-object $X_j$.
However, any two fast-path transactions accessing conflicting data sets must necessarily incur a tracking abort (cf. Remark~\ref{re:traborts}) in $E$. It follows immediately that $S$ must be legal.
\end{proof}
%
% \begin{claim}
% \label{cl:alg1claim}
% %
% If process $p_i$ executing transaction $T_k\in \ms{txns}(E)$ holds the lock on $X_j\in \ms{Wset}(T_k)$ after $E$, then the value $r_j$ is an odd integer value.
% \end{claim}
% %
% \begin{proof}
% %
% \end{proof}
% %
%
\begin{claim}
\label{cl:readfrom}
$S$ is legal, i.e., every t-read returns the value of the latest committed t-write in $S$.
\end{claim}
%
\begin{proof}
%
We claim that for every $\Read_j(X_m) \rightarrow v$, there exists some slow-path transaction $T_i$ (or resp. fast-path)
that performs $\Write_i(X_m,v)$ and completes the event in Line~\ref{line:write} (or resp. Line~\ref{line:lin2}) such that
$\Read_j(X_m) \not\prec_H^{RT} \Write_i(X_m,v)$.

Suppose that $T_i$ is a slow-path transaction:
since $\Read_j(X_m)$ returns the response $v$, the event in Line~\ref{line:read2}
succeeds the event in Line~\ref{line:write} performed by $\TryC_i$. 
Since $\Read_j(X_m)$ can return a non-abort response only after $T_i$ releases the lock on $r_m$ in
Line~\ref{line:rel1}, $T_i$ must be committed in $S$.
Consequently,
$\ell_{\TryC_i} <_E \ell_{\Read_j(X_m)}$.
Since, for any updating
committing transaction $T_i$, $\delta_{T_i}=\ell_{\TryC_i}$, it follows that
$\delta_{T_{i}} <_E \delta_{T_{j}}$.

Otherwise if $T_i$ is a fast-path transaction, then clearly $T_i$ is a committed transaction in $S$.
Recall that $\Read_j(X_m)$ can read $v$ during the event in Line~\ref{line:read2}
only after $T_i$ applies the $\ms{commit-cache}$ primitive.
By the assignment of linearization points, 
$\ell_{\TryC_i} <_E \ell_{\Read_j(X_m)}$ and thus, $\delta_{T_{i}} <_E \ell_{\Read_j(X_m)}$.

Thus, to prove that $S$ is legal, it suffices to show that  
there does not exist a
transaction $T_k$ that returns $C_k$ in $S$ and performs $\Write_k(X_m,v')$; $v'\neq v$ such that $T_i <_S T_k <_S T_j$. 
%

$T_i$ and $T_k$ are both updating transactions that commit. Thus, 
($T_i <_S T_k$) $\Longleftrightarrow$ ($\delta_{T_i} <_{E} \delta_{T_k}$) and
($\delta_{T_i} <_{E} \delta_{T_k}$) $\Longleftrightarrow$ ($\ell_{\TryC_i} <_{E} \ell_{\TryC_k}$).

%
Since, $T_j$ reads the value of $X$ written by $T_i$, one of the following is true:
$\ell_{\TryC_i} <_{E} \ell_{\TryC_k} <_{E} \ell_{\Read_j(X_m)}$ or
$\ell_{\TryC_i} <_{E} \ell_{\Read_j(X_m)} <_{E} \ell_{\TryC_k}$.

Suppose that $\ell_{\TryC_i} <_{E} \ell_{\TryC_k} <_{E} \ell_{\Read_j(X_m)}$.

(\textit{Case \RNum{1}:}) $T_i$ and $T_k$ are slow-path transactions.

Thus, $T_k$ returns a response from the event in Line~\ref{line:acq} 
before the read of the base object associated with $X_m$ by $T_j$ in Line~\ref{line:read2}. 
Since $T_i$ and $T_k$ are both committed in $E$, $T_k$ returns \emph{true} from the event in
Line~\ref{line:acq} only after $T_i$ releases $r_{m}$ in Line~\ref{line:rel1}.

If $T_j$ is a slow-path transaction, 
recall that $\Read_j(X_m)$ checks if $X_j$ is locked by a concurrent transaction, 
then performs read-validation (Line~\ref{line:abort0}) before returning a matching response. 
Indeed, $\Read_j(X_m)$ must return $A_j$ in any such execution.

If $T_j$ is a fast-path transaction, it follows that $\Read_j(X_m)$ must return $A_j$
immediately from Remark~\ref{re:traborts}.

Thus, $\ell_{\TryC_i} <_E \ell_{\Read_j(X)} <_{E} \ell_{\TryC_k}$.

(\textit{Case \RNum{2}:}) $T_i$ is a slow-path transaction and $T_k$ is a fast-path transaction.
Thus, $T_k$ returns $C_k$ 
before the read of the base object associated with $X_m$ by $T_j$ in Line~\ref{line:read2}, but after the response
of \emph{acquire} by $T_i$ in Line~\ref{line:acq}.
Since $\Read_j(X_m)$ reads the value of $X_m$ to be $v$ and not $v'$, $T_i$ performs the \emph{cas}
to $v_m$ in Line~\ref{line:write} after the $T_k$ performs the $\ms{commit-cache}$ primitive (since if
otherwise, $T_k$ would be aborted in $E$).
But then the \emph{cas} on $v_m$ performed by $T_i$ would return $\false$ and $T_i$ would return $A_i$---contradiction.

(\textit{Case \RNum{3}:}) $T_k$ is a slow-path transaction and $T_i$ is a fast-path transaction.
This is analogous to the above case.

(\textit{Case \RNum{4}:}) $T_i$ and $T_k$ are fast-path transactions.
Follows immediately from Claim~\ref{cl:fast}.

We now need to prove that $\delta_{T_{j}}$ indeed precedes $\ell_{\TryC_k}$ in $E$.
Consider the two possible cases.
Suppose that $T_j$ is a read-only transaction. 
Then, $\delta_{T_j}$ is assigned to the last t-read performed by $T_j$ that returns a non-A$_j$ value. 
If $\Read_j(X_m)$ is not the last t-read that returned a non-A$_j$ value, then there exists a $\Read_j(X')$ such that 
$\ell_{\Read_j(X_m)} <_{E} \ell_{\TryC_k} <_E \ell_{read_j(X')}$.
But then this t-read of $X'$ must abort by performing the checks in Line~\ref{line:abort0} or incur a tracking set abort---contradiction.

Otherwise suppose that $T_j$ is an updating transaction that commits, then $\delta_{T_j}=\ell_{\TryC_j}$ which implies that
$\ell_{read_j(X)} <_{E} \ell_{\TryC_k} <_E \ell_{\TryC_j}$. Then, $T_j$ must neccesarily perform the checks
in Line~\ref{line:abort3} and return $A_j$ or incur a tracking set abort---contradiction to the assumption that $T_j$ is a committed transaction.%
%
\end{proof}
%
Since $S$ is legal and respects the real-time ordering of transactions, Algorithm~\ref{alg:inswrite} is opaque.
%
