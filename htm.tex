\documentclass[nocopyrightspace,numbers,10pt,preprint]{sigplanconf}
%\documentclass{sig-alt-release2}
%\documentclass{llncs}
%\usepackage{llncsdoc}
\usepackage{verbatim}
\usepackage{amssymb}
\usepackage{lipsum}
\usepackage{multicol}
\usepackage{amsmath}
\usepackage{framed}
\usepackage{float}
% \floatstyle{boxed} 
 \restylefloat{figure}
\usepackage{subfig}
%\usepackage{mdframed}
%\usepackage{amsthm}
\usepackage{tikz}
%\usepackage[margin=0.99in]{geometry}
\usepackage{hyperref}
\usepackage{color}
\usepackage{soul}
\usepackage{tabularx}
%\usepackage{fullpage}
\usepackage{enumitem}
\setlist{nolistsep}
%\setenumerate{itemsep=0pt}
\usepackage{tikz}
\usepackage{cleveref}
\crefname{section}{\S}{\S\S}
\newif\ifcode
\codefalse
%%%
% comment this if you do not have
% algpseudocode package for algorithms.
\codetrue
%%%
\usepackage{macros}
\ifcode
\usepackage[ruled]{algorithm}
\usepackage{algpseudocode}
\usepackage{ioa_code}

%%%%%%%%%%%%%%%%%%%%%%%%%%%%%%%%%%%%%%%%%%%%%%%%%%%%%%%%%%%%%%%%%%%%%%%%%=
%%%%%%%%%%%%%%%%%%%%%%%%%%
%pour avoir plus de place
% \topmargin 0pt
% \advance \topmargin by -\headheight
% \advance \topmargin by -\headsep
% \textheight 8.9in
% \oddsidemargin 0pt
% \evensidemargin \oddsidemargin
% \marginparwidth 0.5in
% \textwidth 6.5in
% \setlength{\baselineskip}{13.2pt} % standard value is 13.75pt
\def\algofont{\footnotesize} %fonte pour les algos
%\interfootnotelinepenalty=10
%fin de pour avoir plus de place
%du coup il faut commenter la suite
%\textwidth 130mm
%\textheight 215mm
%jusqu'ici
%\renewcommand{\baselinestretch}{1.5}

%%%%%%%%%%%%%%%%%%%%%%%%%%%%%%%%%%%%%%%%%%%%%%%%%%%%%%%%%%%%%%%%
% Definitions
%%%%%%%%%%%%%%%%%%%%%%%%%%%%%%%%%%%%%%%%%%%%%%%%%%%%%%%%%%%%%%%%

\newtheorem{theorem}{Theorem}
%\newtheorem{reptheorem}[]{Theorem \ref{th:sl}}
%\newreptheorem{theorem}{Theorem}
\newtheorem{axiom}[theorem]{Axiom}
\newtheorem{case}[theorem]{Case}
\newtheorem{claim}[theorem]{Claim}
\newtheorem{proposition}{Proposition}
\newtheorem{explanation}[theorem]{Explanation}
\newtheorem{remark}[theorem]{Remark}
\newtheorem{fact}[theorem]{Fact}
% \newtheorem{conclusion}[theorem]{Conclusion}
% \newtheorem{condition}[theorem]{Condition}
\newtheorem{conjecture}[theorem]{Conjecture}
\newtheorem{corollary}[theorem]{Corollary}
% \newtheorem{criterion}[theorem]{Criterion}
\newtheorem{definition}{Definition}
% \newtheorem{exercise}[theorem]{Exercise}
\newtheorem{lemma}[theorem]{Lemma}
% \newtheorem{notation}[theorem]{Notation}
% \newtheorem{problem}[theorem]{Problem}
%\newtheorem{proposition}[theorem]{Proposition}
% \newtheorem{remark}[theorem]{Remark}
% \newtheorem{solution}[theorem]{Solution}
% \newtheorem{summary}[theorem]{Summary}
\newtheorem{observation}[theorem]{Observation}
%\newenvironment{proof}[1][Proof]{\noindent\textbf{#1.} }{\hfill $\Box$\\[2mm]} %\rule{0.5em}{0.5em}\\}
\newenvironment{proof}[1][Proof]{\noindent\textbf{#1.} }{\hfill $\Box$\\[2mm]}
\newenvironment{proofsketch}[1][Proof sketch]{\noindent\textbf{#1.} }{\hfill $\Box$\\[2mm]}


\newenvironment{reptheorem}[1][Theorem]{\noindent\textbf{#1}}{}
\def\lf{\tiny}
\def\rrnnll{\setcounter{linenumber}{0}}
\def\nnll{\refstepcounter{linenumber}\lf\thelinenumber}
\newcounter{linenumber}
\newenvironment{algorithmfig}{\hrule\vskip 3mm}{ \vskip 3mm \hrule }

\def\P{\ensuremath{\mathcal{P}}}
\def\DP{\ensuremath{\Diamond\mathcal{P}}}
\def\DS{\ensuremath{\Diamond\mathcal{S}}}
%\def\T{\ensuremath{\mathcal{T}}}
\def\Time{\mathbb{T}}
\def\S{\ensuremath{\mathcal{S}}}
\def\D{\ensuremath{\mathcal{D}}}
\def\W{\ensuremath{\mathcal{W}}}
\def\A{\ensuremath{\mathcal{A}}}
\def\B{\ensuremath{\mathcal{B}}}
\def\F{\ensuremath{\mathcal{F}}}
\def\R{\ensuremath{\mathcal{R}}}
\def\N{\ensuremath{\mathcal{N}}}
\def\I{\ensuremath{\mathcal{I}}}
\def\O{\ensuremath{\mathcal{O}}}
\def\Q{\ensuremath{\mathcal{Q}}}
\def\K{\ensuremath{\mathcal{K}}}
\def\L{\ensuremath{\mathcal{L}}}
\def\M{\ensuremath{\mathcal{M}}}
\def\V{\ensuremath{\mathcal{V}}}
\def\E{\ensuremath{\mathcal{E}}}
\def\C{\ensuremath{\mathcal{C}}}
\def\T{\ensuremath{\mathcal{T}}}
\def\X{\ensuremath{\mathcal{X}}}
\def\Y{\ensuremath{\mathcal{Y}}}
\def\Nat{\ensuremath{\mathbb{N}}}
\def\Om{\ensuremath{\Omega}}
\def\ve{\varepsilon}
\def\fd{failure detector}
\def\cfd{\ensuremath{?\P+\DS}}
\def\afd{timeless}
\def\env{\ensuremath{\mathcal{E}}}
%\def\faulty{unreliable}
\def\bounded{one-shot}
\def\cons{\textit{cons}}
\def\val{\textit{val}}
\def\code{\textit{code}}

\def\HSS{\mathit{h}}
\def\argmin{\mathit{argmin}}
\def\proper{\mathit{proper}}
\def\content{\mathit{content}}
\def\Level{\mathit{L}}
\def\Blocked{\mathit{Blocked}}
\def\Set{\mathit{Set}}
\def\TS{\mathit{TS}}
\def\shared{\mathit{shared}}
\def\exclusive{\mathit{exclusive}}

\newcommand{\correct}{\mathit{correct}}
\newcommand{\RNum}[1]{\uppercase\expandafter{\romannumeral #1\relax}}
\newcommand	{\faulty}{\mathit{faulty}}
\newcommand{\infi}{\mathit{inf}}
\newcommand{\live}{\mathit{live}}
\newcommand{\true}{\mathit{true}}
\newcommand{\false}{\mathit{false}}
\newcommand{\stable}{\mathit{Stable}}
\newcommand{\setcon}{\mathit{setcon}}
\newcommand{\remove}[1]{}

\newcommand{\Wset}{\textit{Wset}}
\newcommand{\Rset}{\textit{Rset}}
\newcommand{\Dset}{\textit{Dset}}

%\newcommand{\parts}{\textit{parts}}
\newcommand{\txns}{\textit{txns}}

\newcommand{\Read}{\textit{read}}
\newcommand{\Write}{\textit{write}}
\newcommand{\TryC}{\textit{tryC}}
\newcommand{\TryA}{\textit{tryA}}
\newcommand{\ok}{\textit{ok}}

\newcommand{\trylock}{\textit{trylock}}
\newcommand{\multitrylock}{\textit{multi-trylock}}
\newcommand{\CAS}{\textit{CAS}}
\newcommand{\mCAS}{\textit{mCAS}}

\def\Nomega{\ensuremath{\neg\Omega}}
\def\Vomega{\ensuremath{\overrightarrow{\Omega}}}

\newcommand{\id}[1]{\mbox{\textit{#1}}}% for identifiers in code
\newcommand{\res}[1]{\mbox{\textbf{#1}}}% reserved words

\newcommand{\ignore}[1]{}

\definecolor{dkgreen}{rgb}{0,0.6,0}
\definecolor{gray}{rgb}{0.5,0.5,0.5}
\definecolor{mauve}{rgb}{0.58,0,0.82}


\definecolor{gpcolor}{rgb}{0.6,0.2,0.3}
\newboolean{showcomments}
\setboolean{showcomments}{true}
\ifthenelse{\boolean{showcomments}}
{ \newcommand{\mynote}[3]{
    \fbox{\bfseries\sffamily\scriptsize#1}
    {\small$\blacktriangleright$\textsf{\emph{\color{#3}{#2}}}$\blacktriangleleft$}}
}
{ 
\newcommand{\mynote}[3]{}}
\newcommand{\sr}[1]{\mynote{SR}{#1}{magenta}}

\newcommand{\comnospace}{\mbox{$\triangleright$}}
\newcommand{\com}{\mbox{\comnospace\ }}
\usepackage{tabto}
\newcommand{\medcom}[1]{\tabto{5cm} \com \mbox{#1}}
\usepackage{listings}
\lstset{numbers=left,numberblanklines=false}
\usepackage{microtype}
\usepackage[T1]{fontenc}
\usepackage{lmodern}
\usepackage[scaled]{beramono}
\newcommand\Small{\fontsize{7.6}{7.6}\selectfont}
\newcommand*\LSTfont{\Small\ttfamily\SetTracking{encoding=*}{-60}\lsstyle}
\lstdefinestyle{customc}{
    belowcaptionskip=1\baselineskip,
    breaklines=true,
    frame=L,
    xleftmargin=\parindent,
    language=C,
    showstringspaces=false,
    escapeinside={//}{\^^M},
    basicstyle=\LSTfont, %\scriptsize\ttfamily,
    keywordstyle=\bfseries\color{green!40!black},
    commentstyle=\itshape\color{gray!60!black},
    identifierstyle=\color{blue!50!black},
    stringstyle=\color{orange},
    numbers=left,                    % where to put the line-numbers; possible values are (none, left, right)
    numbersep=5pt,                   % how far the line-numbers are from the code
    numberstyle=\tiny\color{black},  % the style that is used for the line-numbers
    otherkeywords={then,word,process_local,type,xbegin,xabort,xend}
}

\begin{document}

\title{Cost of Concurrency in Hybrid Transactional Memory}
\authorinfo{Trevor Brown\and Srivatsan Ravi}
	   {University of Toronto \and Purdue University}
	   
\maketitle

\newcommand{\trevor}[1]{\textbf{[[#1--Trevor]]}}

%%%%%%%%%%%%%%%%%%%%%%%%%%%%%%%%%%%%%%%%%%%%%%%%%%%%%%%%%%%%%%%%%%%%%%%%%%%%%%%%
%
\begin{abstract}
State-of-the-art \emph{software transactional memory (STM)} implementations achieve 
good performance by carefully avoiding the overhead of \emph{incremental validation} 
(i.e., re-reading previously read data items to avoid inconsistency) while
still providing \emph{progressiveness} (allowing transactional aborts only due to \emph{data conflicts}).
Hardware transactional memory (HTM) implementations promise even better performance, 
but offer no progress guarantees.
Thus, they must be combined with STMs, leading to \emph{hybrid} TMs (HyTM)
in which hardware transactions must be \emph{instrumented} (i.e., access metadata) 
to detect contention with software transactions. 

We show that, unlike in progressive STMs, software transactions in progressive HyTMs
cannot avoid incremental validation.
In fact, this result holds even if hardware transactions can \emph{read} metadata 
\emph{non-speculatively}. 
We then present \emph{opaque} HyTM algorithms providing \emph{progressiveness for a subset of transactions} 
that are  optimal in terms of hardware instrumentation. 
We explore the concurrency vs. hardware instrumentation vs. software validation
tradeoffs for these algorithms.
Preliminary experiments with Intel's HTM 
seem to suggest that the inherent \emph{cost to concurrency} 
in HyTMs also exists in practice. Finally, we discuss algorithmic techniques for cicumventing this cost.
\end{abstract}

%\newpage
\pagenumbering{arabic}\setcounter{page}{1}

%!TEX root = htm.tex
\section{Introduction}
\label{sec:intro}
%


\begin{figure*}[!ht]
      
     \scalebox{1}[1]{
     \begin{tabularx}{\textwidth}{c|c|c|c}
%	\hline
	~~~~~ & Algorithm~\ref{alg:inswrite} & Algorithm~\ref{alg:inswrite2} & TLE\\ \hline
	property & No & yes & Yes \\ \hline
	property & No & yes & Yes \\ \hline
	property & No & yes & Yes 
	%    \hline
   \end{tabularx}
\caption{Table}\label{fig:main}    
}
\end{figure*}
%
%!TEX root = htm.tex
\section{Hybrid transactional memory (HyTM)}
\label{sec:hytm}
%
In this section, we extend the formal model of HyTMs originally proposed in \cite{htmdisc15} to accomodate non-speculative accesses within hardware transactions.
The resulting model is sufficiently expressive to capture the executions of existing hardware transactions like \emph{Intel Haswell}~\cite{Rei12},
\emph{IBM Power8}~\cite{bluegene} and \emph{AMD ASF}~\cite{asf}.

\paragraph{Transactional memory (TM).} 
A \emph{transaction} is a sequence of \emph{transactional operations}
(or \emph{t-operations}), reads and writes, performed on a set of \emph{transactional objects} 
(\emph{t-objects}). 
A TM \emph{implementation} provides a set of
concurrent \emph{processes} with deterministic algorithms that implement reads and
writes on t-objects using  a set of \emph{base objects}.
More precisely, for each transaction $T_k$, a TM implementation must support the following t-operations: 
$\mathit{read}_k(X)$, where $X$ is a t-object, that returns a value in
a domain $V$
or a special value $A_k\notin V$ (\emph{abort}),
$\mathit{write}_k(X,v)$, for a value $v \in V$,
that returns $\mathit{ok}$ or $A_k$, and
$\mathit{tryC}_k$ that returns $C_k\notin V$ (\emph{commit}) or $A_k$.
Additionally, we assume that a transaction $T_k$
may perform a $\mathit{start_k}$ that is the first t-operation performed by $T_k$ prior to invoking any $\mathit{read}_k$ or $\mathit{write}_k$.

\paragraph{Configurations and executions.} 
A \emph{configuration} of a TM implementation specifies the state of each base object and each process. 
In the \emph{initial} configuration, each base object has its initial value and each process is in its initial state. 
An \emph{event} (or \emph{step}) of a transaction invoked by some process is an invocation of a t-operation, 
a response of a t-operation, or an atomic \emph{primitive} operation applied to base object along with its response. 
%An event $e$ is \emph{applicable} to a configuration $C$ if $e$ can legally be applied to $C$. 
%Applying an event $e$ to a configuration $C$ results in another configuration $C' = e(C)$.
An \emph{execution fragment} is a (finite or infinite) sequence of events $E = e_1,e_2,\dots$. 
%$E$ is applicable to a configuration $C$ if $e_1$ is applicable to $C$, $e_2$ is applicable to $e_1(C)$, 
%and so forth. 
An \emph{execution} of a TM implementation $\mathcal{M}$ is an
execution fragment where, informally, each event respects the
specification of base objects and the algorithms specified by $\mathcal{M}$.

%applicable to the initial configuration 
%where each event of each transaction is issued according to $\mathcal{M}$.
The \emph{read set} (resp., the \emph{write set}) of a transaction $T_k$ in an execution $E$,
denoted $\Rset_E(T_k)$ (and resp. $\Wset_E(T_k)$), is the set of t-objects that $T_k$ attempts to read (and resp. write) 
by issuing a t-read (and resp. t-write) invocation in $E$ (for brevity, we sometimes 
omit the subscript $E$ from the notation).
The \emph{data set} of $T_k$ is $\Dset(T_k)=\Rset(T_k)\cup\Wset(T_k)$.
$T_k$ is called \emph{read-only} if $\Wset(T_k)=\emptyset$; \emph{write-only} if $\Rset(T_k)=\emptyset$ and
\emph{updating} if $\Wset(T_k)\neq\emptyset$.

For any finite execution $E$ and execution fragment $E'$, $E\cdot E'$ denotes the concatenation of $E$ and $E'$
and we say that $E\cdot E'$ is an \emph{extension}
of $E$.
%Let $E$ be an execution fragment.
For every transaction identifier $k$,
$E|k$ denotes the subsequence of $E$ restricted to events of
transaction $T_k$.
If $E|k$ is non-empty,
we say that $T_k$ \emph{participates} in $E$,
and let $\txns(E)$ denote the set of transactions that participate in $E$.
Two executions $E$ and $E'$
are \emph{indistinguishable} to a set $\mathcal{T}$ of transactions, if
for each transaction $T_k \in \mathcal{T}$, $E|k=E'|k$.

A transaction $T_k\in \txns(E)$ is \emph{complete in $E$} if
$E|k$ ends with a response event.
The execution $E$ is \emph{complete} if all transactions in $\txns(E)$
are complete in $E$.
A transaction $T_k\in \txns(E)$ is \emph{t-complete} if $E|k$
ends with $A_k$ or $C_k$; otherwise, $T_k$ is \emph{t-incomplete}.

We assume that base objects are accessed with \emph{read-modify-write} (rmw) primitives. 
A rmw primitive $\langle g,h \rangle$ applied to a base object 
atomically updates the value of the object with a new value, which is
a function $g(v)$ of the old value $v$, and returns a response $h(v)$.
%A rmw primitive is \emph{trivial} if it never affects the value of a
%base object, otherwise it is \emph{nontrivial}.
A rmw primitive event on a base object is \emph{trivial} if, in any configuration, its application
does not change the state of the object. 
Otherwise, it is called \emph{nontrivial}.
Events $e$ and $e'$ of an execution $E$  \emph{contend} on a base
object $b$ if they are both primitives on $b$ in $E$ and at least 
one of them is nontrivial.

\paragraph{Hybrid transactional memory executions.}
We now describe the execution model of a \emph{Hybrid transactional memory (HyTM)} implementation.
In our HyTM model, shared memory configurations may be modified by accessing base objects via two kinds of
primitives: \emph{direct} and \emph{cached}.
(i) In a direct access, the rmw primitive operates on the memory state:
the direct-access event atomically reads the value of the object in
the shared memory and, if necessary, modifies it.
(ii) In a cached access performed by a process $i$, the rmw primitive operates on the \emph{cached}
state recorded in process $i$'s \emph{tracking set} $\tau_i$ (analogous to \emph{L1 cache} of process $i$). 

More precisely, $\tau_i$ is a set of triples $(b, v, m)$ where $b$ is a base object identifier, $v$ is a value, 
and $m \in \{\shared, \exclusive\}$ is an access \emph{mode}. 
The triple $(b, v, m)$ is added to the tracking set when $i$ performs a cached
rmw access of $b$, where $m$ is set to $\exclusive$ if the access is
nontrivial, and to $\shared$ otherwise.  
%we assume that there exists some constant $\TS$ (representing the size of the L1 cache)
% such that the condition $|\tau_i| \leq \TS$ must always hold; this
% condition will be enforced by our model.
A base object $b$ is \emph{present} in $\tau_i$ with mode $m$ if $\exists v, (b,v,m) \in \tau_i$.

A trivial (resp.\ nontrivial) 
cached primitive $\langle g,h \rangle$ applied to $b$ 
by process $i$ 
% first checks the condition $|\tau_i|=\TS$ and if so, it
% sets $\tau_i=\emptyset$ and immediately returns $\bot$ (we call this event a
% \emph{capacity abort}). 
% We assume that $\TS$ is large enough so that no transaction 
% with data set of size $1$ can incur a capacity abort.
% %
% If the transaction does not incur a capacity abort, the process 
checks whether $b$ is present in exclusive
(resp.\ any) mode in $\tau_j$ 
for any $j\neq i$. If so, $\tau_i$ is set to $\emptyset$ and the
primitive returns $\bot$. 
%
Otherwise, the triple $(b, v, \shared)$ (resp. $(b, g(v), \exclusive)$)
is added to $\tau_i$,  where $v$ is the most recent cached value of $b$ in $\tau_i$
(in case $b$ was previously accessed by $i$ within the current
hardware transaction) or the value of $b$ in the current
memory configuration, and finally $h(v)$ is returned.
%

\paragraph{Hardware aborts.}
A tracking set can be \emph{invalidated} by a concurrent process: 
if, in a configuration $C$ where  $(b,v,\exclusive)\in\tau_i$
(resp.\ $(b,v,\shared)\in\tau_i)$,  a process $j\neq i$ applies any primitive 
(resp.\ any \emph{nontrivial} primitive) to $b$, then $\tau_i$ becomes
\emph{invalid} and any subsequent cached primitive invoked by $i$
sets $\tau_i$ to $\emptyset$ and returns $\bot$. We refer to this event as a \emph{tracking set abort}.

Note that hardware transactions may also abort spuriously, or because of unsupported operations~\cite{Rei12}, or due to \emph{capacity} aborts. 
Since modelling these aborts will not affect our results, except make for a more cumbersome presentation, we only consider capacity aborts in this paper. 

Any transaction $T_k \in \ms{txns}(E)$ that performs at least one cached access necessarily performs a \emph{cache-commit} primitive as the last event of $E|k$. 
A \emph{cache-commit} primitive issued by process $i$ with
a valid $\tau_i$ does the following: for each base object $b$ such that $(b,v,\exclusive) \in \tau_i$, the value of $b$ in $C$ is updated to $v$. 
Finally, $\tau_i$ is set to $\emptyset$ and the primitive 
returns $\textit{commit}$. 

\paragraph{Slow-path and fast-path transactions.}
In the following, we partition HyTM transactions into \emph{fast-path transactions} and \emph{slow-path transactions}.
Practically, two separate algorithms (fast-path one and slow-path one) 
are provided for each t-operation. 

A slow-path transaction models a regular software transaction.
An event of a slow-path transaction is either an invocation or response of a t-operation, or
a direct rmw primitive on a base object. 

A fast-path transaction essentially encapsulates a hardware transaction. Specifically, in any execution $E$,
we say that a transaction $T_k\in \ms{txns}(E)$ is a fast-path transaction if $E|k$ contains at least one cached event.
A events of a \emph{hardware transaction} includes series of direct trivial accesses and at least one cached access
followed by a \emph{cache-commit} primitive.
Note that we specifically allow hardware transactions to perform reads without adding the corresponding base object to
the process's tracking set, thus modelling the non-speculative accesses inside hardware allowed by 
IBM Power8 architectures. We remark that Intel Haswell does not support this feature: an event of a hardware transaction
does not include any direct access.
We assume that a fast-path transaction $T_k$ returns $A_k$
as soon an underlying cached primitive or \emph{cache-commit} returns $\bot$. 
%
%
%
%!TEX root = htm.tex
\subsection{Lower bound}
%
\sr{Result assumes no cached accesses inside fast-path}
%
\begin{lemma}
\label{lm:hytm}
%
LLet $\mathcal{M}$ be any progressive HyTM implementation.
Let $E\cdot E_1 \cdot E_2$ be an execution of $\mathcal{M}$ where
$E_1$ (and resp. $E_2$) is the step contention-free
execution fragment of fast-path transaction $T_1 \not\in \ms{txns}(E)$ (and
resp. $T_2 \not\in \ms{txns}(E)$),
$T_1$ (and resp. $T_2$) does not conflict with any transaction in $E\cdot E_1 \cdot E_2$. 
Then, $T_1$ and $T_2$ do not contend on any base object in $E\cdot E_1 \cdot E_2$~\cite{htmdisc15}.
\end{lemma}

\begin{theorem}
\label{th:impossibility}
There does not exist any progressive opaque HyTM implementation $\mathcal{M}$ with invisible reads such that in every execution $E$ of $\mathcal{M}$, any t-read operation $op_k$ performed by some slow-path transaction $T_k \in \ms{txns}(E)$
performs $O(1)$ steps.
\end{theorem}
%
\begin{proof}
Suppose by contradiction that there exists a progressive opaque HyTM implementation $\mathcal{M}$ with invisible reads such that in every execution $E$ of $\mathcal{M}$, any t-read operation $op_k$ performed by some slow-path transaction $T_k \in \ms{txns}(E)$
performs $\kappa$ steps.

Let $m \in \mathbb{N}$ be a constant such that $m \gg \kappa$ and for all $i\in \{1,\ldots , m\}$, let $v$ be the initial value of t-object $X_i$.
Let $\pi^{m}$ denote the complete step contention-free execution of a slow-path transaction
$T_{\phi}$ that performs ${m}$ t-reads: $\Read_{\phi}(X_1)\cdots \Read_{\phi}(X_{m})$
such that for all $i\in \{1,\ldots , m \}$, $\Read_{\phi}(X_i) \rightarrow v$.
%
\begin{claim}
\label{cl:readdap}
For all $i\in \mathbb{N}$, $\mathcal{M}$ has an execution of the form $\pi^{i-1}\cdot \rho^i\cdot \alpha^i$ where,
%
\begin{itemize}
\item
$\pi^{i-1}$ is the complete step contention-free execution of slow-path read-only transaction $T_{\phi}$ that performs
$(i-1)$ t-reads: $\Read_{\phi}(X_1)\cdots \Read_{\phi}(X_{i-1})$,
\item
$\rho^i$ is the t-complete step contention-free execution of a fast-path transaction $T_{i}$
that writes $nv_i\neq v_i$ to $X_i$ and commits,
\item
$\alpha_i$ is the complete step contention-free execution fragment of $T_{\phi}$ that performs its $i^{th}$ t-read:
$\Read_{\phi}(X_i) \rightarrow nv_i$.
\end{itemize}
%
\end{claim}
%
\begin{proof}
%
$\mathcal{M}$ has an execution of the form $\rho^i\cdot \pi^{i-1}$.
Since $\Dset(T_k) \cap \Dset(T_{i}) =\emptyset$ in $\rho^i\cdot \pi^{i-1}$,
by Lemma~\ref{lm:hytm}, transactions $T_{\phi}$ and $T_i$ do not contend
on any base object in execution $\rho^i\cdot \pi^{i-1}$.
Thus, $\rho^i\cdot \pi^{i-1}$ is also an execution of $M$.

By opacity, $\rho^i\cdot \pi^{i-1} \cdot \alpha_i$ is an execution
of $M$ in which the t-read of $X_i$ performed by $T_{\phi}$ must return $nv_i$.
But $\rho^i \cdot \pi^{i-1} \cdot \alpha_i$ is indistinguishable to $T_{\phi}$ from
$\pi^{i-1}\cdot \rho^i \cdot \alpha_i$.
Thus, $M$ has an execution of the form $\pi^{i-1}\cdot \rho^i \cdot \alpha_i$.
\end{proof}
%
By Claim~\ref{cl:readdap}, for all $i\in \{2,\ldots, m\}$, $M$ has an execution of the form 
$E^{i}=\pi^{i-1}\cdot \rho^i \cdot \alpha_i$.

For each $i\in \{2,\ldots, m\}$, $j\in \{1,2\}$ and $\ell \leq (i-1)$, 
we now define an execution of the form  $\mathbb{E}_{j\ell}^{i}=\pi^{i-1}\cdot \beta^{\ell}\cdot \rho^i \cdot \alpha_j^i$
as follows:
%
\begin{itemize}
\item
%$\rho^m$ is defined as above;
$\beta^{\ell}$ is the t-complete step contention-free execution fragment of a fast-path transaction $T_{\ell}$
that writes $nv_{\ell}\neq v$ to $X_{\ell}$ and commits
\item
$\alpha_1^i$ (and resp. $\alpha_2^i$) is the complete step contention-free execution fragment of 
$\Read_{\phi}(X_i) \rightarrow v$ (and resp. $\Read_{\phi}(X_i) \rightarrow A_{\phi}$).
\end{itemize}
%
\begin{claim}
\label{cl:ic2}
For all $i\in \{2,\ldots, m\}$ and $\ell \leq (i-1)$, $\mathcal{M}$ has an execution of the form $\mathbb{E}_{1\ell}^{i}$ or 
$\mathbb{E}_{2\ell}^{i}$.
\end{claim}
%
\begin{proof}
%
For all $i \in \{2,\ldots, m\}$, $\pi^{i-1}$
is an execution of $\mathcal{M}$.
By assumption of invisible reads, $T_{{\ell}}$ must be committed in $\pi^{i-1}\cdot \rho^{\ell}$
and $M$ has an execution of the form $\pi^{i-1}\cdot \beta^{\ell}$.
By the same reasoning, since $T_i$ and $T_{\ell}$ do not have conflicting data sets,
$M$ has an execution of the form $\pi^{i-1}\cdot\beta^{\ell}\cdot \rho^i$.

Since the configuration after $\pi^{i-1}\cdot\beta^{\ell}\cdot \rho^i$ is quiescent,
$\pi^{i-1}\cdot\beta^{\ell}\cdot \rho^i$ extended with $\Read_{\phi}(X_i)$
must return a matching response.
If $\Read_{\phi}(X_i) \rightarrow v_i$, then clearly $\mathbb{E}_{1}^{i}$
is an execution of $M$ with $T_{\phi}, T_{i-1}, T_i$ being a valid serialization
of transactions.
If $\Read_{\phi}(X_i) \rightarrow A_{\phi}$, the same serialization
justifies an opaque execution.

Suppose by contradiction that there exists an execution of $\mathcal{M}$ such that
$\pi^{i-1}\cdot\beta^{\ell}\cdot \rho^i$ is extended with the complete execution
of $\Read_{\phi}(X_i) \rightarrow r$; $r \not\in \{A_{\phi},v\}$. 
The only plausible case to analyse is when $r=nv$.
Since $\Read_{\phi}(X_i)$ returns the value of $X_i$ updated by $T_i$, 
the only possible serialization for transactions is $T_{\ell}$, $T_i$, $T_{\phi}$; but $\Read_{\phi}(X_{\ell})$
performed by $T_k$ that returns the initial value $v$
is not legal in this serialization---contradiction.
\end{proof}
%
\begin{claim}
\label{cl:three}
For all $i\in \{2,\ldots, m\}$, $j\in \{1,2\}$ and $\ell \leq (i-1)$, transaction $T_{\phi}$ must access
$(i-1)$ different base objects during the execution of $\Read_{\phi}(X_i)$ in the execution
$\pi^{i-1}\cdot \beta^{\ell}\cdot \rho^i \cdot \alpha_j^i$.
\end{claim}
%
\begin{proof}
By the assumption of invisible reads,
the execution $\pi^{i-1}\cdot \beta^{\ell}\cdot \rho^i \cdot \alpha_j^i$
is indistinguishable to
transactions $T_{\ell}$ and $T_{i}$
from the execution ${\tilde \pi}^{i-1}\cdot \beta^{\ell}\cdot \rho^i \cdot \alpha_j^i$, where $\Rset(T_{\phi})=\emptyset$
in ${\tilde \pi}^{i-1}$.
But transactions $T_{\ell}$ and $T_{i}$ access mutually disjoint data sets in ${\tilde \pi}^{i-1}\cdot \beta^{\ell}\cdot \rho^i$ and by Lemma~\ref{cl:hytm},
they cannot contend on the same base object in this execution.

Consider the $(i-1)$ different executions: 
$\pi^{i-1}\cdot\beta^{1}\cdot \rho^i$, $\ldots$, $\pi^{i-1}\cdot\beta^{i-1}\cdot \rho^i$.
For all $\ell, \ell' \leq (i-1)$;$\ell' \neq \ell$, 
$M$ has an execution of the form $\pi^{i-1}\cdot \beta^{\ell}\cdot \rho^i \cdot \beta^{\ell'}$
in which transactions $T_{\ell}$ and $T_{\ell'}$ access mutually disjoint data sets.
By invisible reads and Lemma~\ref{lm:hytm}, the pairs of transactions $T_{\ell'}$, $T_{i}$ and $T_{\ell'}$, $T_{\ell}$
do not contend on any base object in this execution.
This implies that $\pi^{i-1}\cdot \beta^{\ell} \cdot \beta^{\ell'} \cdot \rho^i$ is an execution of $M$ in which
transactions $T_{\ell}$ and $T_{\ell'}$ each apply nontrivial primitives
to mutually disjoint sets of base objects in the execution fragments $\beta^{\ell}$ and $\beta^{\ell'}$ respectively.

This implies that for any $j\in \{1,2\}$, $\ell \leq (i-1)$, the configuration $C^i$ after $E^i$ differs from the configurations
after $\mathbb{E}_{j\ell}^{i}$ only in the states of the base objects that are accessed in the fragment $\beta^{\ell}$.
Consequently, transaction $T_{\phi}$ must access at least $i-1$ different base objects
in the execution fragment $\pi_j^i$
to distinguish configuration $C^i$ from the configurations
that result after the $(i-1)$ different executions 
$\pi^{i-1}\cdot\beta^{1}\cdot \rho^i$, $\ldots$, $\pi^{i-1}\cdot\beta^{i-1}\cdot \rho^i$ respectively.

Thus, for all $i \in \{2,\ldots, m\}$, transaction $T_{\phi}$ must perform at least $i-1$ steps 
while executing the $i^{th}$ t-read in $\pi_{j}^i$.
\end{proof}
%
However, this contradicts the assumption that in every execution $E$ of $\mathcal{M}$, any t-read operation $op_k$ performed by some slow-path transaction $T_k \in \ms{txns}(E)$
performs $\kappa$ steps. This completes the proof.
\end{proof}


%
%!TEX root = htm.tex
\section{Hybrid transactional memory algorithms}\label{sec:hytmalgos}
%
%
In this section, we first identify some invariants that follow from the HyTM execution model presented in the previous section.
We then describe two HyTM implementations: \cref{sec:hytm1} describes a progressive opaque TM while \cref{sec:hytm2} describes an opaque HyTM that is progressive only for read-only slow-path transactions.

Let $E$ be any execution of a HyTM implementation $\mathcal{M}$ in
which a fast-path transaction $T_k$ is either
t-incomplete or aborted. Let $E'$ be the execution fragment that is a subsequence of $E$ derived by removing all cached events of $E|k$.
%
\begin{observation} 
\label{ob:one}
To every slow-path transaction $T_m \in \ms{txns}(E)$, $E$ is indistinguishable from $E'$. Additionally, if a fast-path transaction $T_m\in \ms{txns}(E) \setminus \{T_k\}$ does not incur a tracking set abort in $E$, 
then $E$ is indistinguishable to $T_m$ from $E'$.
\end{observation}
%
%
%%%%%%%%%%%%%%%%%%%%%%%%%%%%%%%%%%%%%%%%%%%%%%%%%%%%%%%%%%%%%%%%%%%%%
%!TEX root = htm.tex
In this section, we present opaque HyTM algorithms that are progressive for a subset of transactions.

\vspace{1mm}\noindent\textbf{Instrumentation-optimal progressive HyTM.}
We describe a HyTM algorithm that is a tight bound for Theorem~\ref{th:impossibility} and the instrumentation cost on the fast-path transactions established in \cite{hytm14disc}.
For every t-object $X_j$, our implementation maintains a base object $v_j$ that stores the value of $X_j$
and a \emph{sequence lock} $r_{j}$. The sequence lock is an unsigned integer whose LSB bit stores the \emph{locked} state.
Specifically, we say that process $p_i$ \emph{holds a lock on $X_j$ after an execution $E$} if
$\textit{or}_j$ $\mathrel{\&} 1=1$ after $E$, where $\textit{or}_j$ is the value of $r_j$ after $E$.

\vspace{1mm}\noindent\textit{Fast-path transactions:}
For a fast-path transaction $T_k$ executed by process $p_i$, the $\Read_k(X_j)$ implementation first reads $r_j$ (direct)
and returns $A_k$ if some other process $p_j$ holds a lock on $X_j$.
Otherwise, it returns the value of $X_j$.
As with $\Read_k(X_j)$, the $\Write (X_j,v)$ implementation returns $A_k$ if some other process $p_j$ holds a lock on $X_j$.
Process $p_i$ then increments the value of $r_j$ by $2$ via a direct access and stores the cached state of $X_j$ along with its value $v$.
If the cache has not been invalidated, $p_i$ updates the shared memory
during $\TryC_k$ by invoking the $\ms{commit-cache}$ primitive.

\vspace{1mm}\noindent\textit{Slow-path read-only transactions:}
Any $\Read_k(X_j)$ invoked by a slow-path transaction first reads the value of the object from $v_j$, 
adds $r_j$ to $\Rset(T_k)$ if its not held by a concurrent transaction
and then performs \emph{validation} on its entire read set to check if any of them have been modified. 
If either of these conditions is true,
the transaction returns $A_k$. Otherwise, it returns the value of $X_j$. 
Validation of the read set is performed by re-reading the values of the sequence lock entires stored in $\Rset(T_k)$.
%A read-only transaction simply returns $C_k$ during the tryCommit.

\vspace{1mm}\noindent\textit{Slow-path updating transactions:}
% The $\Write_k(X,v)$ implementation of a slow-path transaction stores
% $v$ and the current value of $X_j$ locally, 
% deferring the actual update in shared memory to tryCommit. 
An updating slow-path transaction $T_k$ attempts to obtain exclusive write access to its 
entire write set by performing \emph{compare-and-set} (\emph{cas})
primitive that checks if the value of $r_j$, for each $X_j\in \Wset(T_k)$, is unchanged since last reading it during $\Write_k(X.v)$
If all the locks on the write set were acquired successfully, $T_k$ performs validation of the read set and returns $C_k$ if successful, else $p_i$ aborts the transaction.

\vspace{1mm}\noindent\textit{Direct accesses inside fast-path:}
As indicated in the pseudocode of Algorithm~\ref{alg:inswrite}, some accesses may be performed uncached (as allowed
in IBM Power 8) and the resulting implementation would still be opaque. 

\vspace{1mm}\noindent\textbf{Instrumentation-optimal HyTMs that are progressive only for slow-path transactions.}
Algorithm~\ref{alg:inswrite2} does not incur the linear instrumentation cost
on the fast-path reading transactions (as in Algorithm~\ref{alg:inswrite}, but provides progressiveness only
for slow-path reading transactions. The instrumentation cost on fast-path t-reads is avoided by using a global single-bit lock $L$ that serializes all updating slow-path transactions.%

\vspace{1mm}\noindent\textbf{Sacrificing progressiveness and minimizing contention window.}
Observe that the lower bound in Theorem~\ref{th:impossibility} assumes progressiveness for both slow-path and fast-path transactions
along with opacity and invisible reads.
Figure~\ref{fig:main} summarizes the complexity costs
associated with the HyTM algorithms considered in this paper.

Hybrid NOrec~\cite{hybridnorec} is a HyTM implementation that does not satisfy progressiveness
(unlike its STM counterpart NOrec), but mitigates
the step-complexity cost on slow-path transactions by performing incremental validation 
during a transactional read \emph{iff} 
the shared memory has changed since the start of the transaction.
Conceptually, hybrid NOrec uses a global sequence lock \emph{gsl} that is incremented 
at the start and end of each transaction's commit procedure.
Readers can use the value of gsl to determine whether shared memory has changed between two configurations.
Unfortunately, with this approach, two fast path transactions will always conflict on the gsl if their 
commit procedures are concurrent.
To reduce the contention window for fast path transactions, the gsl is actually implemented as two separate locks (the second one called \emph{esl}).
A slow path transaction locks both esl and gsl while it is committing.
Instead of incrementing gsl, a fast path transaction checks if esl is locked and aborts if it is.
Then, at the end of the fast path transaction's commit procedure, 
it increments gsl twice (quickly locking and releasing it and immediately commits in hardware), thus, the 
window for fast path transactions to contend on gsl is very small.
%
%
%
%
%
%!TEX root = htm.tex
\section{Evaluation}
%
%!TEX root = htm.tex
\section{Related work and discussion}
\label{sec:rel}
%
%\vspace{1mm}\noindent\textbf{HyTM complexity.}
The proof of Theorem~\ref{th:impossibility} is based on the analogous proof for step complexity of STMs that are \emph{disjoint-access parallel}~\cite{prog15-pact}.
%Our model extends the HyTM model in \cite{hytm14disc} which did not allow uncached accesses on the fast-path.
%\vspace{1mm}\noindent\textbf{HyTM implementations.}
% An early HyTM implementation described in \cite{damronhytm} uses the \emph{ROCK} processor~\cite{rock} as the underlying HTM
% while \cite{kumarhytm} described an implementation that requires support for \emph{non-cached accesses} in a hardware transaction. 
Early HyTMs like the ones described in \cite{damronhytm, kumarhytm} provided progressiveness, but
subsequent HyTM proposals sacrificed progressiveness for lesser instrumentation overheads.
Recent work has investigated fallback to \emph{reduced} hardware transactions~\cite{MS13}
in which an all-software slow-path is replaced by a mix of hardware and software transactions. 
% Afek \emph{et al}. proposed amalgamated lock elision (ALE)~\cite{ale15} which improves over TLE
% by executing the slow-path as a series of segments, each of which is a dynamic length hardware transaction.
Our implementation of Hybrid NOrec follows \cite{hynorecriegel}, which additionally proposed the use of non-speculative accesses
in fast-path transactions to reduce instrumentation overhead. %(supported in the AMD ASF architectures).

%\vspace{1mm}\noindent\textbf{Concluding remarks.}
In ongoing work, we are implementing our algorithms on the IBM POWER8 HTM implementation which supports
non-cached accesses in hardware transactions.
We hope to understand whether the instrumentation overheads we observed
on Intel's HTM are also inherent to POWER8's HTM implementation.

To our knowledge, ours is the first work to consider the theoretical foundations of the cost of concurrency in HyTMs.
In order to achieve high performance in practice, one must either identify a new progress condition to replace progressiveness or develop a new HyTM algorithm that effectively uses non-speculative writes.
Both directions are promising, and little work has been done in either in the context of today's HTMs.

%%!TEX root = htm.tex
\section{Concluding remarks}
\label{sec:conc}
%
\newpage
%
\bibliographystyle{abbrv}
\bibliography{references2}
\end{document}
%%% Local Variables:
%%% mode: latex
%%% mode: flyspell
%%% Local IspellDict: "american"
%%% mode: outline-minor
%%% End:

