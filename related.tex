%!TEX root = htm.tex
\section{Related work}
\label{sec:rel}
%
\paragraph{HyTM complexity bounds.}

\paragraph{HyTM algorithms.}
Herlihy and Moss introduced \emph{hardware transactional memory (HTM)}
and showed that \emph{bounded-size} atomic transactions could be supported in hardware
with simple modifications to the cache mechanism of existing processors~\cite{HM93}.
To overcome the limitations of bounded HTMs, there have been proposals for ``unbounded HTMs''
that allow transactions to commit even if they exceed the hardware resources~\cite{hammondhytm,unboundedhtm1}.
Damron \emph{et al.}~\cite{damronhytm} and Kumar \emph{et al.}~\cite{kumarhytm} proposed HyTM designs
that integrated HTMs with a variant of \emph{DSTM}~\cite{HLM+03} in which
every t-read operation incurs at least one extra access to a metadata base object, just as Algorithm~\ref{alg:inswrite}.

HyTM implementations like \emph{PhTM}~\cite{phasedtm} and Hybrid NOrec~\cite{hybridnorec}
employ a single metadata check at the start of a hardware transaction to allow
hardware transactions to abort due to \emph{any} concurrent non-conflicting software writer.
Detailed coverage on HyTM implementation design can be found in \cite{HLR10}
and \cite{riegel-thesis}.

Alternatives to STM fallback, such as sandboxing~\cite{ALM14,CTGM14}.
fallback to \emph{reduced} hardware transactions~\cite{MS13}. 

