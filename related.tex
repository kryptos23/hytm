%!TEX root = htm.tex
\section{Related work and discussion}
\label{sec:rel}
%
%\vspace{1mm}\noindent\textbf{HyTM complexity.}
The proof of Theorem~\ref{th:impossibility} is heavily based on the analogous proof for step complexity of
STMs that are \emph{disjoint-access parallel}~\cite{prog15-pact}.
The model from \cref{sec:hytm} itself is an extension of the HyTM model introduced in \cite{hytm14disc}
which does not allow uncached accesses within the fast-path.

%\vspace{1mm}\noindent\textbf{HyTM implementations.}
An early HyTM implementation described in \cite{damronhytm} uses the \emph{ROCK} processor~\cite{rock} as the underlying HTM
while \cite{kumarhytm} described a specific HTM design that requires the support of \emph{non-cached accesses}
within a hardware transaction. 
Recent work has investigated fallback to \emph{reduced} hardware transactions~\cite{MS13}
in which an all-software slow-path is replaced by a mix of hardware and software transactions. 
Afek \emph{et al}. proposed amalgamated lock elision (ALE)~\cite{ale15} which improves over TLE
by executing the slow-path as a series segments, each segment being a dynamic length hardware transaction.
The hybrid NOrec algorithm is described in \cite{hynorecriegel} which additionally proposed the use of non-speculative accesses
within fast-path transactions supported in the AMD ASF architectures.

%\vspace{1mm}\noindent\textbf{Concluding remarks.}
In ongoing work, we are implementing our algorithms on the IBM Power8 HTM implementation which supports
non-cached accesses inside fast-path. We hope to understand if the instrumentation overheads we observed
on Intel's HTM are also inherent to Power8's HTM implementation.