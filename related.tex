%!TEX root = htm.tex
\section{Related work}
\label{sec:rel}
%
\paragraph{HyTM complexity bounds.}
The proof of Theorem~\ref{th:impossibility} is heavily based on the analogous proof for step complexity of
STMs that are \emph{disjoint-access parallel} from \cite{prog15-pact}.
The model from \cref{sec:hytm} itself is an extension of the HyTM model introduced in \cite{hytm14disc}
which does not allow uncached accesses within the fast-path.

\paragraph{HyTM algorithms.}
The HyTM implementation in \cite{damronhytm} uses the \emph{ROCK} processor~\cite{rock} as the underlying HTM
while \cite{kumarhytm} described a specific HTM design that requires the support of \emph{non-cached accesses}
within a hardware transaction. 
Recent work has investigated alternatives to STM fallback, such as sandboxing~\cite{ALM14,CTGM14}, and fallback to \emph{reduced} hardware transactions~\cite{MS13}. These proposals are not currently covered by our framework, although we believe that our model can be extended to incorporate such techniques.
Afeak \emph{et al}. proposed amalgamated lock elision (ALE)~\cite{ale15} which improves over TLE
by executing the slow-path as a series segments, each segment being a dynamic length hardware transaction.
