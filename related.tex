%!TEX root = htm.tex
\section{Related work and discussion}
\label{sec:rel}
%
\vspace{1mm}\noindent\textbf{HyTM implementations.}
Early HyTMs like the ones described in \cite{damronhytm, kumarhytm} provided progressiveness, but
subsequent HyTM proposals sacrificed progressiveness for lesser instrumentation overheads.

To our knowledge, ours is the first work to consider the theoretical foundations of the cost of concurrency in HyTMs.

\vspace{1mm}\noindent\textbf{HyTM complexity.}
The proof of Theorem~\ref{th:impossibility} is based on the analogous proof for step complexity of STMs that are \emph{disjoint-access parallel}~\cite{prog15-pact}.
Our model extends the HyTM model in \cite{hytm14disc} which did not allow direct accesses on the fast-path.

\vspace{1mm}\noindent\textbf{Direct accesses inside hardware.}
\cite{kumarhytm} described an implementation that requires support for direct accesses in a hardware transaction. 
Our implementation of Hybrid NOrec follows \cite{hynorecriegel}, which additionally proposed the use of direct accesses
in fast-path transactions to reduce instrumentation overhead. %(supported in the AMD ASF architectures).

\vspace{1mm}\noindent\textbf{Beyond the two path HyTM approach.}
\vspace{1mm}\noindent\textit{Employing an uninstrumented fast fast-path.}
We now describe how every transaction may first be executed in a ``fast'' fast-path with almost no instrumentation
and if unsuccessful, may be re-attempted in the fast-path and subsequently in slow-path.
Specifically, we transform any opaque HyTM $\mathcal{M}$ to an opaque
HyTM $\mathcal{M}'$ in which a shared \emph{fetch-and-add} metadata base object $F$ that slow-path updating transactions
increment (and resp. decrement) at the start (and resp. end). In $\mathcal{M}'$, a ``fast'' fast-path transaction checks first checks if $F$ is $0$
and if not, aborts the transaction; otherwise the transaction is continued as an uninstrumented hardware transaction.
The code for the fast-path and the slow-path is identical to $\mathcal{M}$.
Assuming the hardware transactions do not perform any direct accesses, opacity is immediate.

Recent work has investigated fallback to \emph{reduced} hardware transactions~\cite{MS13}
in which an all-software slow-path is replaced by a mix of hardware and software transactions. 
Afek \emph{et al}. proposed amalgamated lock elision (ALE)~\cite{ale15} which improves over TLE
by executing the slow-path as a series of segments, each of which is a dynamic length hardware transaction.
