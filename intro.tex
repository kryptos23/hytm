%!TEX root = htm.tex
\section{Introduction}
\label{sec:intro}
%
%
The \emph{Transactional Memory (TM)} abstraction is a synchronization mechanism 
that allows the programmer to \emph{speculatively} execute sequences of shared-memory
operations as \emph{atomic transactions}.
Most popular TM designs, subsequent to the original proposal in \cite{HM93} 
have implemented all the functionality in software~\cite{norec, ST95,HLM+03, astm, fraser}.
The original dynamic STM implementation \emph{DSTM}~\cite{HLM+03} ensures \emph{progressiveness}: 
a transaction aborts only if there is a read-write \emph{data conflict} with a concurrent
transaction. However, read operations in DSTM must \emph{incrementally} validate
every next read operation. This results in a quadratic  (in the size of the transaction's read
set) step-complexity bound. Subsequent STM 
implementations like \emph{NOrec}~\cite{norec} and \emph{TL2}~\cite{DSS06}
minimize the impact on performance due to incremental validation.
NOrec uses a global versioned lock that is read at the start of a transaction and performs \emph{value-based}
validation during read operations only if the value of the global lock has changed since reading it.
TL2 eliminates incremental validation completely by associating a global versioned lock
value with each data item updated: any read of a data item returns
a value only if the version associated with it is the same as the 
value of the versioned lock read at the start of the transaction. 

In fact, state-of-the-art STM implementations ensure progress in the absence of conflicts with 
$O$(1) complexity read operations while still satisfying \emph{invisible reads}.
Intuitively, this means that read-only transactions do not
apply any nontrivial primitives on the shared memory which is considered importance for performance in
read-dominated workloads.

Nonetheless, TM designs that are implemented entirely in software still incur significant performance overhead.
Thus, current CPUs have included instructions to mark a block of memory accesses as transactional~\cite{Rei12, asf, bluegene}, allowing them to be executed \emph{atomically} in hardware.
Hardware transactions promise even better performance, but they offer no progress guarantees 
since they may experience %\emph{conflict aborts} and
\emph{spurious} aborts. This motivates the need for
\emph{hybrid} TMs in which the \emph{fast} hardware transactions are 
complemented with \emph{slower} software transactions that do not experience spurious aborts.

To allow hardware transactions in a HyTM to detect conflicts with software transactions, 
hardware transactions must be \emph{instrumented} to perform additional metadata accesses, which introduces overhead.
Hardware transactions typically provide automatic contention detection for every memory location accessed within one,
thus ensuring that the transaction itself would be aborted.
This is at least the case with the Intel Haswell's Transactional Synchronization Extensions~\cite{haswell}.
The IBM Power8 ISA additionally allows hardware transactions to access metadata \emph{non-speculatively}, thus precluding
contention detection for \emph{all} memory locations accessed. While this has the advantage of potentially reducing contention aborts
in hardware, this makes the design of HyTM implementations potentially harder to prove correct.

In \cite{htmdisc15}, it was shown that hardware transactions in progressive HyTMs must perform
an additional metadata access per transactional operation.
In this paper, we show that in progressive HyTMs, 
software transactions \textit{cannot} avoid incremental validation.
Specifically, we prove that read operations of software transactions in progressive HyTMs
must necessarily incur a validation cost that is \emph{linear} 
in the size of the transaction's read set. 
This is in stark contrast to progressive STMs which can achieve $O(1)$ complexity read operations.
Thus, in addition to the linear instrumentation cost on the hardware transactions, there is the quadratic
step complexity cost on the software transactions.

We then present \emph{opaque} HyTM algorithms providing \emph{progressiveness for a subset of transactions} that are  %both hardware and software transactions \trevor{maybe just ``transactions''?}
optimal in terms of hardware instrumentation. We show 
how \emph{some} hardware instrumentation can be performed \textit{non-speculatively} without violating opacity while exploring the concurrency vs. hardware instrumentation vs. software validation
tradeoffs for these algorithms.
Preliminary experiments on Intel Haswell (resp. IBM Power 8) not supporting (resp. supporting) non-speculative accesses inside hardware,
seem to suggest that the inherent \emph{cost to concurrency} in HyTMs also exists in practice and discuss algorithmic techniques to overcome them.

\paragraph{Roadmap.}
\cref{sec:hytm} presents details of the HyTM model that extends the model introduced in \cite{htmdisc15} 
to access memory locations non-speculatively.
\cref{sec:lb} presents our main lower bound result on the step-complexity of progressive HyTMs
while \cref{sec:hytmalgos} presents several opaque HyTM algorithms.
\cref{sec:eval} summarizes resuls from our preliminary experiments on Intel Haswell and IBM Power 8 architectures.
\cref{sec:rel} presents the related work and \cref{sec:conc} concludes the paper.
%