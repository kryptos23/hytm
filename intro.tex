%!TEX root = htm.tex
\section{Introduction}
\label{sec:intro}
%
The \emph{Transactional Memory (TM)} abstraction is a synchronization mechanism 
that allows the programmer to \emph{speculatively} execute sequences of shared-memory
operations as \emph{atomic transactions}.
Most popular TM designs, subsequent to the original proposal in \cite{HM93} 
have implemented all the functionality in software~\cite{norec, ST95,HLM+03, astm, fraser}.
However, TM designs are implemented entirely in software which typically incurs significant performance overhead.
Thus, current CPUs have included instructions to mark a block of memory accesses as transactional~\cite{Rei12, asf, bluegene}, allowing them to be executed \emph{atomically} in hardware.
However, hardware transactions may be spuriously aborted due to several reasons: cache capacity overflows, interrupts etc., thus leading to \emph{hybrid} TMs 
in which the \emph{fast} hardware transactions are complemented with \emph{slower} software transactions that do not experience spurious aborts.
To allow hardware transactions in a HyTM to detect conflicts with software transactions, hardware transactions must be \emph{instrumented} to perform additional metadata accesses, which introduces overhead.

\begin{figure*}[!ht]
      
     \scalebox{1}[1]{
     \begin{tabularx}{\textwidth}{c|c|c|c|c}
%	\hline
	~~~~~ & Algorithm~\ref{alg:inswrite} & Algorithm~\ref{alg:inswrite2} & TLE & HybridNorec\\ \hline
	Metadata accesses in read-only h/w & No & yes & Yes & Yes \\ \hline
	Metadata accesses in read-only s/f & No & yes & Yes & Yes \\ \hline
	Metadata accesses in updating h/w & No & yes & Yes & Yes \\ \hline
	h/w-s/f concurrency & No & yes & Yes & Yes \\ \hline
	opacity & No & yes & Yes & Yes \\  \hline
	%    \hline
   \end{tabularx}
\caption{Table}\label{fig:main}    
}
\end{figure*}